\documentclass[12pt]{report}
\usepackage[thinc]{esdiff} % for typesettign derivatives
\usepackage{amsthm} % provides an enhanced version of LaTex's \newtheorem command
\usepackage{mdframed} % framed environments that can split at page boundaries
\usepackage{enumitem} % bulletin points or other means of listing things
\usepackage{amsmath} % different results from various cases
\usepackage{amssymb} % for AMS symbols


\theoremstyle{definition}
\mdfdefinestyle{defEnv}{%
  hidealllines=false,
  nobreak=true,
  innertopmargin=-1ex,
}

\pagestyle{headings}
\author{Demitrios PapaGeorgiou}
\title{MATH40011 Calculus for JMC}
\begin{document}
\maketitle
\tableofcontents

\chapter{Differentiation}

\section{Continuity and differentiability}

\newmdtheoremenv[style=defEnv]{theorem}{Definition}
\begin{theorem}
    $f(x)$ is said to be continuous on an interval $[a,b]$ if \[
        \lim_{x \rightarrow x_0} f(x) = f(x_0) \quad \forall x_0 \in [a,b]
    \]
\end{theorem}

\newmdtheoremenv[style=defEnv]{differentiability}[theorem]{Definition}
\begin{differentiability}
    The function $f(x)$ is differentiable at $x$ if \[
        \lim_{h \rightarrow 0} \frac{f(x+h) - f(x)}{h}
    \] (Newton quotient) exists. We call this $f'(x)$ the derivative of $f$ at point $x$.
    \\In other words, a function is differentiable at $x$ 
    if right and left derivatives exists \emph{AND} are equal.\ i.e. \[
        \lim_{t \rightarrow x^{+}} \frac{f(t) - f(x^+)}{t - x^+} 
        = \lim_{t \rightarrow x^{-}} \frac{f(t) - f(x^-)}{t - x^-}
    \]
\end{differentiability}

\newmdtheoremenv[style=defEnv]{differentiable means continuous}[theorem]{Theorem}
\begin{differentiable means continuous}
    If $f(x)$ is differentiable at $x = x_0$ then it is also continuous there.
\end{differentiable means continuous}
\begin{proof}
    \begin{eqnarray*}
        \lim_{x \rightarrow x_0} \Big(f(x) - f(x_0)\big)
        &=& \lim_{x \rightarrow x_0} \left(\frac{f(x) - f(x_0)}{x - x_0} \cdot (x - x_0)\right) \\
        &=& \lim_{x \rightarrow x_0} \left(\frac{f(x) - f(x_0)}{x - x_0}\right) 
        \lim_{x \rightarrow x_0} (x - x_0) \\
        &=& f'(x_0) \cdot 0 \\
        &=& 0
    \end{eqnarray*}
\end{proof}

\newmdtheoremenv[style=defEnv]{max point in an interval}[theorem]{Theorem}
\begin{max point in an interval}
    Let $f$ be a function which is defined and differentiable on the open interval $(a,b)$.
    Let $c$ be a number in the interval which is a maximum for the function. Then $f'(c) = 0$.
    Same if $c$ is a minimum of $f$.
\end{max point in an interval}
\begin{proof}
    $f(c) \ge f(c+h) \Rightarrow f(c+h) - f(c) \le 0$, i.e.\[
        \lim_{h \rightarrow 0^{+}} \frac{f(c+h) - f(c)}{h} \le 0
    \]Similarly for the left limit, \[
    \lim_{h \rightarrow 0^{-}} \frac{f(c+h) - f(c)}{h} \ge 0
\]In order to be differentiable, $f'(c)$ can only be 0.
\end{proof}

\newmdtheoremenv[style=defEnv]{max and min in an interval}[theorem]{Theorem}
\begin{max and min in an interval}
    Let $f(x)$ be continuous on the closed interval $[a,b]$. 
    Then $f(x)$ has a maximum and a minimum on this interval.\
    i.e. $\exists c_1, c_2$ s.t. $f(c_1) \ge f(x)$ and $f(c_2) \le f(x) \;\forall x \in [a,b]$.
\end{max and min in an interval}

\newmdtheoremenv[style=defEnv]{MVT}[theorem]{(Mean Value) Theorem}
\begin{MVT}
    If $f$ is continuous on $[a,b]$ and differentiable on $(a,b)$, 
    then $\exists a < c < b$ s.t. \[
        f'(c) = \frac{f(b) - f(a)}{b-a}
    \]
\end{MVT}

\section{Inverse functions}

\newmdtheoremenv[style=defEnv]{inverse function}[theorem]{Definition}
\begin{inverse function}
    Let $y = f(x)$ be defined on some interval. Given any $y_0$ in the range of $f$, if we can find a \emph{unique}
    value $x_0$ in its domain such that $f(x_0) = y_0$, then we can define the \emph{inverse function}\[
        x = g(y) ( = f^{-1}(y) )
    \]
\end{inverse function}

\newmdtheoremenv[style=defEnv]{strictly increasing function has inverse}[theorem]{Theorem}
\begin{strictly increasing function has inverse}
    Let $f(x)$ be strictly increasing or strictly decreasing. Then the inverse function exists.
\end{strictly increasing function has inverse}

\newmdtheoremenv[style=defEnv]{derivative of inverse f}[theorem]{Theorem}
\begin{derivative of inverse f}
    let $f(x)$ be differentiable on $(a,b)$
    and $f'(x) > 0$ or $f'(x) < 0 \forall x \in (a,b)$.
    Then the inverse function exists and we have\[
        g'(y) ( = f^{-1}(y) ) = \frac{1}{f(x)}
    \]
\end{derivative of inverse f}




\chapter{Integration}
\section{Preliminaries}
    The \emph{anti-derivative} or \emph{integral} of a function $f(x)$
    
    Given $f(x)$ defined over some interval, then if I can find a function $F(x)$ defined over the same interval
    s.t. \[F'(x) = f(x)\]

    Then $F(x)$ is the \emph{indefinite integral} of f, $F = \int f(x) \mathrm{d}x$

    This is not unique.
    Let $G$ be another indefinite integral, i.e. $G'(x) = f(x)$. Then $\diff {(F-G)}{x} = 0 \Rightarrow F(x)=G(x)+K$, 
    where $K$ is a constant.

\subsection{Area under a curve}
Suppose $f(x)\ge0$ in some given interval $[a,b]$ and it is also continuous on $[a,b]$. $(a<b)$ 

Define $F(x)$ to be the area under the curve between $x=a$ and some $x$. By definition $F(a)=0$

\newmdtheoremenv[style=defEnv]{1st part of FTC}[theorem]{Theorem}
\begin{1st part of FTC}
    The function $F(x)$ is differentiable and its derivative is equal to $f(x)$.
    Another way to state this is \[
        \frac{\mathrm{d}}{\mathrm{d}x} \int_a^x {f(t)} \mathrm{d}t = f(x)
    \]That is, the rate of change of the area under curve at $x$ is precisely $f(x)$.
    This is also known as \emph{the first part of Fundamental Theorem of Calculus}.
\end{1st part of FTC}
\begin{proof}
    Newton quotient
    \[
        \frac{F(x+h)-F(x)}{h}
    \]
    Suppose $x\neb$ and also $h>0$.
    $F(x+h)-F(x)$ is the are under the graph between $x$ and $x+h$.

    Since $f(x)$ is continuous on $[x,x+h]$ and is defined in the interval, it must have a maximum at some point $x_+$
    and minimum at some point $x_-$. Hence, $\forall t\in[x,x+h]$, 
    \[
        f(x_-) \le f(t) \le f(x_+)
    \]
    Can also bound the area using the rectangles
    \[
        h\cdot f(x_-) \le F(x+h)-F(x) \le h\cdot f(x_+)
    ,\] i.e.
    \[
        f(x_-) \le \frac{F(x+h)-F(x)}{h} \le f(x_+)
    \]
    Since $x_+$ and $x_-$ are contained in $[x,x+h]$, as $h \to 0$, $x_-, x_+ \to x$ and by the squeezing theorem,
    we have \[
        \lim_{h \to 0} \frac{F(x+h)-F(x)}{h} = f(x)
    ,\] i.e. \[
      F'(x) = f(x)
    \]
    Hence the antiderivative is connected to area under the curve. The constant is fixed by $F(a) = 0$.

    In other words, if I can guess a function $G(x)$ whose derivative is $f(x)$ 
    [e.g. Guess $\log{x}$ for the antiderivative of $\frac{1}{x}$.]
    Then since $F$ and $G$ differ by a constant, we have \[
        F(x) = G(x) + K
    \]
    But $F(a)=0 \Rightarrow -G(a) = K \Rightarrow F(x) = G(x) - G(a)$.
    Hence \[
        \int_a^b f(x)\mathrm{d}x = F(b) = G(b) - G(a)
    \]
    \\This is the familiar \emph{definite integral}.
    This is also known as \emph{the second part of the Fundamental Theorem of Calculus}.
\end{proof}
\newtheorem{ex}[theorem]{Example}
\begin{ex}
    \[
        \int_1^2 x^2\mathrm{d}x = \frac{x^3}{3} \Big|_1^2 = \frac{8}{3} - \frac{1}{3} = \frac{7}{3}
    \,.\]
    Here $f(x) = x^2$, $G(x) = \frac{x^3}{3}$ is the guessed antiderivative.
\end{ex}
\subsection{Signed area definition}
If $f(x)<0$, then area is below the $x$-axis. Define $F(x)$ to be \emph{minus} the area, leading to the definite integral.
\[
    \int_a^b f(x)\mathrm{d}x = F(b) - F(a)
\]
\section{The Riemann Sum}
Given $f(x)$, $a \le x \le b$, take a \emph{partition} of the interval $[a,b]$ to be \[
    x_i = a + ih, i = 0,1,2,\ldots,n
\]\[
    h = \frac{b-a}{n}
\]
\underline{Note}: My partition has regular spacing. Can generalize this to have a partition defined by a sequence 
${\{x_k\}}_{k=0,1,\ldots,n}$ and in the limit $\max\limits_{k}\mathopen|x_k - x_{k-1}\mathclose| \rightarrow 0$.
I am avoiding this technical issue which is quite irrelevant to what we want to do!

Take any subinterval $[x_{i-1}, x_i]$ and let $x^*_i \in [x_{i-1}, x_i]$. Then the Riemann sum is $\sum_{i=1}^n f(x_i^*)h$.

Three particularly useful ways: \[
    x^*_i = x_i - \textnormal{``right-hand RS''}
\]\[
    x^*_i = x_{i-1} - \textnormal{``left-hand RS''}
\]\[
    x^*_i = \frac{1}{2}(x_i + x_{i-1}) - \textnormal{``midpoint RS''}
\]

Now in the limit $n \rightarrow \infty$, $h \rightarrow 0$, we can prove \[
    \lim_{n \rightarrow \infty} \sum_{i=1}^n f(x_i^*)h = \int_a^b f(x) \mathrm{d}x
\]
\emph{Sketch of the proof}: \[
    \textnormal{(Lower Riemann Sum)} \; L_n := \sum_{i=1}^n {\sup f([x_{i-1},x_i])} h
\]\[
    \textnormal{(Upper Riemann Sum)} \; U_n := \sum_{i=1}^n {\inf f([x_{i-1},x_i])} h
\]
By geometry, \[
    L_n \le \int_a^b f(x)\mathrm{d}x \le U_n
\]
In the limit it gets squeezed, if the limit exists then it is the integral.

\section{Properties of the definite integral}
\begin{equation}
    \int_a^b c\,f(x)\mathrm{d}x = c\int_a^b f(x)\mathrm{d}x
\end{equation}
\begin{equation}
    \int_a^b (f(x) + g(x))\mathrm{d}x = \int_a^b f(x)\mathrm{d}x + \int_a^b g(x)\mathrm{d}x
\end{equation}
If $c\in(a,b)$ (and here $a<b$), then
\begin{equation}
    \int_a^b f(x)\mathrm{d}x = \int_a^c f(x)\mathrm{d}x + \int_c^b f(x)\mathrm{d}x
\end{equation}
If $f(x) \le g(x)$ for $x \in [a,b]$, then
\begin{equation}
    \int_a^b f(x)\mathrm{d}x \le \int_a^b g(x)\mathrm{d}x
\end{equation}
(Hence $\int_a^b f(x)\mathrm{d}x \le \int_a^b |f(x)|\mathrm{d}x$ 
and $|\int_a^b f(x)\mathrm{d}x| \le \int_a^b |f(x)|\mathrm{d}x$)
\begin{equation}
    \int_b^a f(x)\mathrm{d}x = -\int_a^b f(x)\mathrm{d}x
\end{equation}

\subsection{Fundamental Theorem of Calculus}
Suppose $F$ is differentiable on $[a,b]$ and $F'$ is integrable on $[a,b]$. Then \[
    \int_a^b F'(x)\mathrm{d}x = F(b) - F(a)
\] If $f$ is integrable on $[a,b]$ and has \emph{antiderivative} $F$, then \[
    \int_a^b f(x)\mathrm{d}x = F(b) - F(a)
\]
\newmdtheoremenv[style=defEnv]{FTC}[theorem]{Theorem}
\begin{FTC}
    \[
        \frac{\mathrm{d}}{\mathrm{d}x}\int_a^{g(x)} f(t)\mathrm{d}t = f(g(x))\cdot g'(x)
    \]
\end{FTC}
\begin{proof}
    Let \[
        F(x) = \int_a^x f(t)\mathrm{d}t
    \]
    Then $F'(x) = f(x)$ --- already proved.
    Now $\int_a^{g(x)} f(t)\mathrm{d}t = F(g(x))$ by definition of $F$.
    \begin{eqnarray*}
            \Rightarrow \frac{\mathrm{d}}{\mathrm{d}x} \int_a^{g(x)}f(t)\mathrm{d}t 
            &=& \frac{\mathrm{d}}{\mathrm{d}x} F(g(x)) \\
            &=& F'(g(x)) \cdot g'(x) \\
            &=& f(g(x)) \cdot g'(x)
    \end{eqnarray*}
\end{proof}
\begin{ex}
    \[
        \frac{\mathrm{d}}{\mathrm{d}x} \int_a^{x^2} e^t \mathrm{d}t = e^{x^{2}} \cdot 2x 
    \]\[
    \textnormal{or } \int_a^{x^{2}} e^{t}\mathrm{d}t = e^{t} |_a^{x^{2}} = e^{x^{2}} - e^{a} \textnormal{ (same as before)}
    \]
\end{ex}

\section{Improper Integrals}
\newmdtheoremenv[style=defEnv]{improper integral}[theorem]{Definition}
\begin{improper integral}
    $\int_a^{b} f(x) \mathrm{d}x$ is an \emph{improper integral} if
    \begin{enumerate}[label = (\roman*)]
        \item $a = \infty$ and/or $b = \infty$
        \item $f(x) \rightarrow \pm \infty$ in $(a,b)$
    \end{enumerate}
\end{improper integral}
To find improper integrals, we take the limit of proper integrals. If the limit is finite, the integral \emph{converges};
otherwise it \emph{diverges}.
\begin{ex}
    \,

    \begin{enumerate}[label = (\roman*)]
        \item $\int_1^{\infty} \frac{1}{x^{2}} \mathrm{d}x 
            = \lim_{b\rightarrow \infty} \int_1^{b} \frac{1}{x^{2}} \mathrm{d}x 
            = \lim_{b\rightarrow\infty} (-\frac{1}{b} + 1) = 1$
        \item $\int_1^{\infty} \frac{\mathrm{d}x}{x} = \lim_{b \rightarrow \infty} \log{b} = \infty$, \,i.e.\ diverges.
    \end{enumerate}
\end{ex}

\subsection{Comparison Test}
\fbox{\begin{minipage}{\linewidth}
Suppose $f$ and $g$ satisfy
\begin{enumerate}[label = (\arabic*)]
    \item $f(x) \le g(x)\; \forall x \ge a$
    \item $\exists \int_a^{b} f(x)\mathrm{d}x \textnormal{ and } \int_a^{b} g(x)\mathrm{d}x \;\forall b > a$
\end{enumerate}
Then 
\begin{enumerate}[label = (\roman*)]
    \item If $\int_a^{\infty} g(x)\mathrm{d}x$ is convergent, so is $\int_a^{\infty} f(x)\mathrm{d}x$
    \item If $\int_a^{\infty} f(x)\mathrm{d}x$ is divergent, so is $\int_a^{\infty} g(x)\mathrm{d}x$
\end{enumerate}
Similarly for $\int_{-\infty}^{b} f(x)\mathrm{d}x$ and $\int_{-\infty}^{\infty} f(x)\mathrm{d}x$
\end{minipage}}
\,

Comparison test is useful if we cannot carry out the integral exactly. It will tell us if it exists, 
then we can find it numerically etc.

\begin{flushleft}
Common tricks for evaluation:
\begin{enumerate}
    \item Substitute constants with unknowns, or the other way round. 
        The most common variables are $x$ and $\cos{x}$ or $\sin{x}$. 
        Omitting/Adding certain variables may be better in some cases!
    \item Substitute an expression with another expression which can be smaller/bigger for successful evaluation.
        Use derivative if necessary.
    \item (\textit{Special}) Use whatever means necessary (such as Integration by Parts) to generate $x^{p}$, 
        where $p \le -1$, so that the integral converges.
\end{enumerate}
\end{flushleft}

\begin{ex}
    \;

    \begin{enumerate}[label = (\arabic*)]
        \item \[
                % the warning here is very awkward,
                % why do I have to wrap around () with {}?
                \int_0^{\infty} \frac{\sin{x}}{{(1+x)}^2} \mathrm{d}x \textnormal{ converges}
        \]
        First thing to quote is that $\int_0^{\infty} \frac{\mathrm{d}x}{{(1+x)}^{2}}$ converges by comparison to
        $\int_1^{\infty} \frac{\mathrm{d}x}{x^{2}}$. (Why? If $x \ge 1$, $\frac{1}{{(1+x)}^{2}} < \frac{1}{x^{2}}$.)

        By the comparison test, it converges because \[
        \frac{|\sin{x}|}{{(1+x)}^{2}} \le \frac{1}{{(1+x)}^{2}} < \frac{1}{x^{2}} \textnormal{ for } x \ge 1
        \]
        \item \[
                \int_1^{\infty} \frac{\mathrm{d}x}{\sqrt{1+x^{2}}} \textnormal{ diverges}
        \]\[
        \int_1^{b} \frac{\mathrm{d}x}{\sqrt{1+x^{2}}} \ge \int_1^{b} \frac{\mathrm{d}x}{\sqrt{x^2 + x^{2}}}
                = \int_1^{b} \frac{\mathrm{d}x}{\sqrt{2}x}
        \]
        Now $\int_1^{\infty} \frac{\mathrm{d}x}{x}$ diverges 
        $\Rightarrow$ So does $\int_1^{\infty} \frac{\mathrm{d}x}{\sqrt{1+x^{2}}}$

        \item \[
                \int_1^{\infty} \frac{\mathrm{d}x}{\sqrt{x}} \textnormal{ diverges}
        \]
        Here is a proof using the comparison theorem.

        If $x \ge 1$, \[
            \frac{1}{x} \le \frac{1}{\sqrt{x}} 
            \Rightarrow \int_1^{b} \frac{1}{x} \mathrm{d}x < \int_1^{b} \frac{\mathrm{d}x}{\sqrt{x}}, \;b > 1
        \] and $\int_1^{\infty} \frac{1}{x} \mathrm{d}x$ diverges.
    \end{enumerate}
    
\end{ex}

\subsection{Improper integrals of unbounded functions}
WLOG, consider situation where $|f(x)| \rightarrow \infty$ as $x \rightarrow 0$. Again take limits of bounded integrals.
E.g. \[
    \int_0^{1} \frac{1}{x^{p}}\mathrm{d}x 
    \begin{cases}
        \textnormal{converges} & \textnormal{if } p < 1\\
        \textnormal{diverges} & \textnormal{if } p \ge 1%
    \end{cases}
\]
\begin{proof}
    Exercise!
\end{proof}

\begin{ex}
    \,

    \begin{enumerate}[label = (\arabic*)]
        \item 
            \begin{eqnarray*}
                \int_0^{1} \log{x} \mathrm{d}x 
                &=& \lim_{\varepsilon \rightarrow 0} \int_\varepsilon^{1} \log{x} \mathrm{d}x \\
                &=& \lim_{\varepsilon \rightarrow 0} (x\log{x}|_\varepsilon^{1} - 
                \int_\varepsilon^{1} x \frac{1}{x}\mathrm{d}x) \\
                &=& \lim_{\varepsilon \rightarrow 0} (-\varepsilon\log{\varepsilon} - 1 + \varepsilon) \\
                &=& -1
            \end{eqnarray*}
        \item Show that the improper integral $I = \int_0^{\infty} \frac{e^{-x}}{\sqrt{x}}\mathrm{d}x$ converges.

            Write $I = I_1 + I_2$ where $I_1 = \int_0^{1} \frac{e^{-x}}{\sqrt{x}} \mathrm{d}x$, 
            $I_2 = \int_1^{\infty} \frac{e^{-x}}{\sqrt{x}} \mathrm{d}x$.\[
                I_1 = \int_0^{1} \frac{e^{-x}}{\sqrt{x}}\mathrm{d}x < \int_0^{1} \frac{1}{\sqrt{x}} \mathrm{d}x
                \textnormal{, which is convergent,}
            \]\[
            \int_1^{\infty} \frac{e^{-x}}{\sqrt{x}} \mathrm{d}x < \int_1^{\infty} e^{-x} \mathrm{d}x
            \textnormal{, which is also convergent.}
            \]
    \end{enumerate}
\end{ex}

\subsection{Mean Value Theorem for Integrals}
Given a function $f$ that is integrable on $[a,b]$, we define its average $\langle f \rangle_{[a,b]}$ by the formula \[
    \langle f(x) \rangle_{[a,b]} = \frac{1}{b-a} \int_a^{b} f(x)\mathrm{d}x
\]
Since $\langle f \rangle_{[a,b]}$ is a number (constant), then we have \[
    \int_a^{b} f(x) \mathrm{d}x = \int_a^{b} \langle f \rangle_{[a,b]} \mathrm{d}x
\]

\newmdtheoremenv[style=defEnv]{MVT for Integrals}[theorem]{Theorem}
\begin{MVT for Integrals}
    Let $f$ be continuous on $[a,b]$. Then $\exists\, x_0 \in (a,b)$ s.t. \[
        f(x_0) = \frac{1}{b-a} \int_a^{b} f(x) \mathrm{d}x
    \]
\end{MVT for Integrals}
\begin{proof}
    Define $F(x) = \int_a^{x} f(t)\mathrm{d}t$. By the FTC we have \[
    F'(x) = f(x) \; \forall x \in (a,b)
    \]
    $F$ is continuous at $a$ and $b$ (proof in exercises). By MVT we have \[
        F'(x_0) = \frac{F(b) - F(a)}{b - a}
    \]i.e.\[
    f(x_0) = \frac{\int_a^{b}f(t)\mathrm{d}t - \int_a^{a}f(t)\mathrm{d}t}{b - a} = \frac{1}{b-a} \int_a^{b}f(t)\mathrm{d}t
    \]
\end{proof}

\newmdtheoremenv[style=defEnv]{Generalized MVT for Integrals}[theorem]{Theorem}
\begin{Generalized MVT for Integrals}
    let $f$ and $g$ be continuous on $[a,b]$ with $g(x) \ge 0 \,\forall x \in [a,b]$. Then $\exists \,c\in [a,b]$ with \[
        \int_a^{b} f(x)g(x)\mathrm{d}x = f(c)\int_a^{b} g(x)\mathrm{d}x
    \]
\end{Generalized MVT for Integrals}
\begin{proof}
    Since $f$ is continuous on $[a,b]$ it must have maximum $M$ and a minimum $m$ on $[a,b]$, i.e. $m \le f(x) \le M$.
    Since $g(x) \ge 0$, we have $\forall \, x \in [a,b]$, \[
        mg(x) \le f(x)g(x) \le Mg(x)
    \]
    By the properties of integral, we have \[
        m\int_a^{b} g(x)\mathrm{d}x \le \int_a^{b} f(x)g(x)\mathrm{d}x \le M\int_a^{b} g(x)\mathrm{d}x
    \]
    By IVT, $\exists \, c \in [a,b]$ s.t. \[
        f(c)\int_a^{b} g(x)\mathrm{d}x = \int_a^{b} f(x)g(x)\mathrm{d}x
    \]
\end{proof}

\section{Applications of Integration}
\subsection{Arc Length}
\begin{eqnarray*}
    \textnormal{Total length } L &\approx & \sum_{i=1}^{n} \sqrt{{(x_i - x_{i-1})}^{2} + {(f(x_i) - f(x_{i-1}))}^{2}} \\
                                 &=& \sum_{i=1}^{n} (x_i - x_{i-1}) 
                                 \sqrt{1 + {\left(\frac{f(x_i)-f(x_{i-1})}{x_i-x_{i-1}}\right)}^{2}}
\end{eqnarray*}
Now let $x_i - x_{i-1} = h = \frac{b-a}{n} := \Delta x$,
\begin{eqnarray*}
    L &=& \lim_{n \rightarrow \infty} 
      \sum_{i=1}^{n} \Delta x \sqrt{1 + {\left(\frac{f(x_i) - f(x_{i-1})}{x_i - x_{i-1}}\right)}^{2}} \\
      &=& \int_a^{b} {\left(1+{(f'(x))}^{2}\right)}^{\frac{1}{2}} \mathrm{d}x
\end{eqnarray*}
In parametric form this is \[
    L = \int_{t_0}^{t_1} {\left[{\left(\frac{\mathrm{d}x}{\mathrm{d}t}\right)}^{2} + 
    {\left(\frac{\mathrm{d}y}{\mathrm{d}t}\right)}^{2} \right]}^{\frac{1}{2}} \mathrm{d}t
\]
\subsection{Volumes and surface area of revolution}
Given an area bounded by $x=a$, $x=b$, $y=f(x)$, $y=0$, then the volume of the solid produced by revolving $y=f$
about the $x$-axis is given by\[
    V = \int_a^{b} \pi {(f(x))}^{2} \mathrm{d}x
\]
Revolving the element about the $y$-axis gives a shell of volume \[
    V = \int_a^{b} 2\pi x f(x) \mathrm{d}x
\]
As we revolve about the $x$-axis, the area of the surface area swept out is a strip of length $\approx 2\pi f(x_i)$
and thickness
\begin{eqnarray*}
    \Delta l_i &= & {[{(x_i - x_{i-1})}^{2} + {(f(x_i) - f(x_{i-1}))}^{2}]}^{\frac{1}{2}} \\
               &\approx & {[1 + {{(f'(x_i))}^{2}}]}^{\frac{1}{2}} \Delta x
\end{eqnarray*}
$\Rightarrow$ in the limit, area $S$ is \[
    S = \int_a^{b} 2\pi f(x) \sqrt{1 + {(f'(x))}^{2}} \mathrm{d}x
\]
\subsection{Centre of mass}
We must have a $0$ total moment in each axis. For instance, in 2D:\@\[
    \left.
    \begin{array}{rl}
        \text{(1)} & \sum m_i(\overline{x} - x_i) = 0 \\
        \text{(2)} & \sum m_i(\overline{y} - y_i) = 0
    \end{array}
    \right \} \Rightarrow(\overline{x}, \overline{y}) 
    = \left(\frac{\sum\ m_i x_i}{\sum\ m_i}, \frac{\sum\ m_i y_i}{\sum\ m_i} \right) 
\]
\fbox{\begin{minipage}{\linewidth}
\emph{General theory} --- divide into small rectangles, and so the moment about the whole plate $A$ about the $y$-axis is\[ \iint_A x\rho(x,y)\mathrm{d}x\mathrm{d}y = \overline{x}\iint_A \rho(x,y)\mathrm{d}x\mathrm{d}y
\]
Similarly, about the $x$-axis,\[
    \iint_A y\rho(x,y)\mathrm{d}x\mathrm{d}y = \overline{y}\iint_A \rho(x,y)\mathrm{d}x\mathrm{d}y
\]
\end{minipage}}
\\\\In the case of having a region $\{(x,y): a \le x \le b, 0 \le y \le f(x)\}$,\[
    m = \int_a^b \rho f(x) \mathrm{d}x
\]\[
    M_y = \int_a^{b} \rho xf(x)\mathrm{d}x
\]\[
    M_x = \frac{1}{2}\int_a^{b} \rho {(f(x))}^{2} \mathrm{d}x
\]Then\[
\overline{x} = \frac{\int_a^{b} xf(x)\mathrm{d}x}{\int_a^{b} f(x)\mathrm{d}x}
\qquad
\overline{y} = \frac{\frac{1}{2} \int_a^{b} {(f(x))}^{2}\mathrm{d}x}{\int_a^{b} f(x)\mathrm{d}x}
\]
In the case of having a region $\{(x,y): a \le x \le b, g(x) \le y \le f(x)\}$, similarly\[
    \overline{x} = \frac{\int_a^{b} x(f(x) - g(x))\mathrm{d}x}{\int_a^{b} f(x) - g(x) \mathrm{d}x}
    \qquad
    \overline{y} = \frac{\frac{1}{2} \int_a^{b} {(f(x))}^{2} - {(g(x))}^{2} \mathrm{d}x}
    {\int_a^{b} f(x) - g(x) \mathrm{d}x}
\]

\subsection{Theorem of Pappus}
Let $R$ be a region that lies on one side of a line $l$.
\begin{itemize}
    \item $A$ &= area of $R$
    \item $V$ &= Volume obtained by rotating about $l$
    \item $d$ &= distance travelled by the centre of mass when $R$ is rotated about $l$
\end{itemize}
Then\[
    V = Ad
\]
\begin{ex}
    Volume of a cylinder radius $r$
    \\Take the function $y = r$, $0 \le x \le l$. Rotate about $x$-axis.\[
        A = rl
    \]
    since $\overline{y} = \frac{1}{2}r$ (symmetry), \[
        \Rightarrow d = \frac{1}{2} r \cdot 2\pi = \pi r
    \]\[
        V = rl \cdot \pi r = \pi r^{2} l \textnormal{ as known}
    \]
\end{ex}

\section{Length and area in polar coordinates}
\subsection{Length}
\begin{description}
    \item[\textit{Approach 1}] \,
            \\\\Recall\[
                L = \int_a^{b} {\left[{\left(\frac{\mathrm{d} x}{\mathrm{d}t} \right)}^{2}
                + {\left(\frac{\mathrm{d} y}{\mathrm{d}t} \right)}^{2}\right]}^{\frac{1}{2}}
            \]for parametric curves $(x(t),y(t))$.
            Now in polar coordinates we have curves $r = f(\theta)$. So use $\theta$ as a parameter,\[
                x = r\cos{\theta} = f(\theta)\cos{\theta}, \quad y = r\sin{\theta} = f(\theta)\sin{\theta}
            \]Substitute $x$ and $y$ into $L$ above, we get
            \begin{eqnarray*}
                L &=& \int_{\alpha}^{\beta} {\left[{(f'\cos{\theta}-f\sin{\theta})}^{2} 
                  + {(f'\sin{\theta}+f\cos{\theta})}^{2}\right]}^{\frac{1}{2}}\mathrm{d} \theta \\
                  &=& \int_\alpha ^{\beta} {\left[{\left(f'(\theta)\right)}^{2} 
                  + {\left(f(\theta)\right)}^{2}\right]}^{\frac{1}{2}} \mathrm{d}\theta \\
                  &=& \int_\alpha ^{\beta} {\left[{\left(\frac{\mathrm{d} r}{\mathrm{d}\theta} \right)}^{2} 
                  + r^{2}\right]}^{\frac{1}{2}}\mathrm{d}\theta
            \end{eqnarray*}
        \item[\textit{Approach 2}] \,
            \\\\Use Pythagoras theorem to write infinitessimals,\[
                {(\mathrm{d}r)}^{2} + r^{2}{(\mathrm{d}\theta)}^{2} = {(\mathrm{d}s)}^{2},
                \quad
                \mathrm{d}s = {\left[{\left(\frac{\mathrm{d}r}{\mathrm{d}\theta}\right)}^{2} + r^{2}\right]}^{\frac{1}{2}}
            \]\[
            \Rightarrow L = \int_\alpha ^{\beta} {\left[{\left(\frac{\mathrm{d}r}{\mathrm{d}\theta}\right)}^{2} 
            + r^{2}\right]}^{\frac{1}{2}} \mathrm{d}\theta
            \]
\end{description}
\begin{ex}
    Find the length of the cardioid $r = 1 + \cos{\theta}$, $0 \le \theta \le 2\pi$.
    \begin{eqnarray*}
        L &=& \int_0^{2\pi}\sqrt{{\left(1+\cos{\theta}\right)}^{2} + \sin^{2}\theta} \mathrm{d}\theta \\
          &=& \int_0^{2\pi} \sqrt{2 + 2\cos(\theta)} \mathrm{d}\theta \\
          &=& \int_0^{\pi} 2\cos{\frac{\theta}{2}}\mathrm{d}\theta
          + \int_\pi^{2\pi} -2\cos{\frac{\theta}{2}}\mathrm{d}\theta \quad
          \text{\textcolor{red}{Be careful with this step!}} \\
          &=& 8
    \end{eqnarray*}
\end{ex}

\subsection{Area}
Using segments of angles $\Delta \theta_i$,\[
    \Delta A = \frac{1}{2} {\left(f(\theta)\right)}^{2} \Delta \theta_i
\]\[
\Rightarrow A = \frac{1}{2}\int_\alpha^{\beta} {\left(f(\theta)\right)}^{2}\mathrm{d}\theta 
= \frac{1}{2}\int_\alpha^{\beta} r^{2}\mathrm{d}\theta
\]
\begin{ex}
    Find the area enclosed by the four-petaled rose $r = \cos{2\theta}$.
    \begin{eqnarray*}
        A &=& \frac{1}{2}\int_{-\frac{\pi}{4}}^{\frac{\pi}{4}} \cos^{2}{2\theta}\,\mathrm{d}\theta \\
          &=& \frac{1}{2}\int_{-\frac{\pi}{4}}^{\frac{\pi}{4}} \frac{1+\cos{4\theta}}{2}\mathrm{d}\theta \\
          &=& \left(\frac{1}{4}\right)\left(\frac{\pi}{2}\right) \\
          &=& \frac{\pi}{8}
    \end{eqnarray*}
    
\end{ex}

\chapter{Power Series and Taylor's Theorem}
\section{Power Series}
\subsection{Convergence tests and radius of convergence}
\newmdtheoremenv[style=defEnv]{power series}[theorem]{Definition}
\begin{power series}    
    Let $x$ be a real number (can extend to complex numbers also) and ${\{a_{n}\}}_{n \ge 0}$ be a sequence of numbers.
    Then we can form the \emph{power series} $\sum_{n=0}^{\infty} a_n x^{n}$.
    The partial sums $S_N = \sum_{n=1}^{N} a_n x^{n}$ are degree $N$ polynomials.
\end{power series}
\newmdtheoremenv[style=defEnv]{radius of convergence}[theorem]{(Radius of Convergence) Theorem}
\begin{radius of convergence}
    $\exists R \in [0,\infty]$ s.t.
    \begin{itemize}
            \item $|z| < R \Rightarrow \sum a_n z^{n}$ is absolutely convergent, and
            \item $|z| > R \Rightarrow \sum a_n z^{n}$ is divergent
    \end{itemize}
\end{radius of convergence}
\begin{proof}
    Let $S = \{|z|: a_n z^{n} \rightarrow 0\}$, nonempty since $0 \in S$. Then define \[
        R = 
        \begin{cases}
            \sup{S} & \text{if $S$ bounded,} \\
            \infty & \text{if $S$ unbounded.}
        \end{cases}
    \]Now suppose $|z| < R$.
    Since $|z|$ is not an upper bound for $S$, $\exists w$ s.t. $|w| > |z|$ and $a_w w^{n} \rightarrow 0$.
    In particular $|a_n w^{n}|$ is bounded by some $A \;\forall n$. Thus \[
        |a_n z^{n}| = |a_n w^{n}|{\left|\frac{z}{w}\right|}^{n} \le A {\left|\frac{z}{w}\right|}^{n}
    \]
    Therefore, by comparison with the convergent series $\sum {\left|\frac{z}{w}\right|}^{n}$ 
    (recall $\left|\frac{z}{w}\right| < 1$)
    we find $\sum |a_n z^{n}|$ is convergent.

    On the other hand, if $|z| > R$ then $a_n z^{n} \nrightarrow 0$ as $n \rightarrow \infty \Rightarrow 
    \sum a_n z^{n}$ diverges.
\end{proof}
\subsection{Differentiation and integration of power series}
\newmdtheoremenv[style=defEnv]{diff and int of power series}[theorem]{Theorem}
\begin{diff and int of power series}
    Let $f(x) = \sum_{n=0}^{\infty} a_n x^{n}$ be a power series which converges absolutely for $|x| < R$.
    \\Then for $|x| < R$, $f(x)$ is differentiable\[
        f'(x) = \sum_{n=1}^{\infty} n a_n x^{n-1}
    \]and integrable\[
    \int f(x)\mathrm{d}x = \sum_{n=0}^{\infty} \frac{a_n x^{n+1}}{n+1}
    \]
    For a power series \emph{within its radius of convergence}, we can differentiate or integrate term by term.
\end{diff and int of power series}
\begin{notes}
    $f(x) = \sum_{n=0}^{\infty} a_n x^{n}$ can be differentiated an infinite number of times as long as $|x| < R$, 
    and the derivatives will exist.
    The way to show this is to consider each differentiated series as a new power series. For example,
    \begin{eqnarray*}
        \frac{\mathrm{d}^{k}}{\mathrm{d}x^{k}} \left(\sum_{}^{} a_n x^{n}\right)
        &=& \sum_{}^{} n(n-1)(n-2)\cdots(n-(k-1))x^{(n-k)}a_n \\
        &=& \sum_{}^{} \frac{n!}{(n-k)!} a_n x^{n-k}
    \end{eqnarray*}
    Using ratio test,\[
        \lim_{n\rightarrow\infty} \frac{(n+1)!}{(n+1-k)!} \frac{(n-k)!}{n!} \left|\frac{a_{n+1}}{a_n}\right| |x|
        = \lim_{n\rightarrow\infty} \left(\frac{n+1}{n+1-k}\right) \left|\frac{a_{n+1}}{a_n}\right| |x|
        = L|x|
    \]as for the undifferentiated power series,\[
    \Rightarrow \frac{\mathrm{d}^{k}}{\mathrm{d}x^{k}} \left(\sum_{}^{} a_n x^{n}\right) 
    \textnormal{converges for} \;|x| < R
    \]
\end{notes}

\begin{ex}
    Write down power series for $\frac{x}{1+x^{2}}$ and $\log{(1+x^{2})}$
    \\\\Recall geometric series $1+r+r^{2}+\cdots = \frac{1}{1-r}$ for $|r| < 1$.
    If $r = -x^{2}$ we have $x(1-x^{2}+x^{4}-x^{6}+\cdots) = \frac{1}{1+x^{2}} \cdot x$. Hence\[
        \frac{x}{1+x^{2}} = \sum_{n=0}^{\infty} {(-1)}^{n}x^{2n+1}\text{,} \qquad |x| < 1 \;\text{for convergence}
    \]Now for the $\log{(1+x^{2})}$ we observe that \[
    \frac{\mathrm{d}}{\mathrm{d}x} \log{(1+x^{2})} = \frac{2x}{1+x^{2}} %chktex 29
    \]So\[
    \log{(1+x^{2})} = 2\int \frac{x}{1+x^{2}} \mathrm{d}x
    \]If $|x| < 1$ we can integrate term by term.\,i.e.
    \begin{eqnarray*}
        \log{(1+x^{2})} &=& 2\int (x - x^{3} + x^{5} - \cdots)\mathrm{d}x \\
                        &=& x^{2} - \frac{x^{4}}{2} + \frac{x^{6}}{3} - \frac{x^{8}}{4} + \cdots
    \end{eqnarray*}
\end{ex}
\newmdtheoremenv[style=defEnv]{power series algebraic operations}[theorem]{Theorem}
\begin{power series algebraic operations}
    Let $f(x) := \sum_{n=0}^{\infty} a_n x^{n}$ and $g(x) := \sum_{n=0}^{\infty} b_n x^{n}$,
    with radius of convergence being $R_1$ and $R_2$. Let $R:=\min{R_1, R_2}$. Then for $|x| < R$,
    \begin{enumerate}[label = (\arabic*)]
        \item $f(x) + g(x) = \sum_{n=0}^{\infty} (a_n + b_n)x^{n}$
        \item $cf(x) = \sum_{n=0}^{\infty} c\,a_n x^{n}$
        \item $f(x)g(x) = \sum_{n=0}^{\infty} \left(\sum_{m=0}^{n} a_m b_{n-m}\right) x^{n}$
    \end{enumerate}
\end{power series algebraic operations}

\section{Taylor Series}
This is a power series that represents a function $f(x)$ by using its derivatives at a single point.

Assume the power series exists and identifies the coefficients. Take a fixed point $x = x_0$.
If $\sum_{n=0}^{\infty} a_n {(x-x_0)}^n$ converges for $|x-x_0|$ small enough, we can find the coefficients as\[
    a_k = \frac{1}{k!}f^{(k)}(x_0)
\]\[
\Rightarrow f(x) = \sum_{n=0}^{\infty} \frac{f^{(n)}(x_0)}{n!}{(x-x_0)}^{n}
\]This is the \emph{Taylor series about the point $x=x_0$}. If $x_0=0$ we get the \emph{Maclaurin series} \[
f(x) = \sum_{n=0}^{\infty} \frac{f^{(n)}(0)}{n!} x^{n} 
\]In both formulas we have assumed that $f(x)$ is infinitely differentiable on some interval containig the point $x=x_0$.
This of course can happen for many functions, e.g. $f(x) = \sin{x}$, $f(x) = e^{x}$, $f(x) = \cos{x}, \ldots$

\subsection{Derivation of power series}
\begin{enumerate}
    \item By definition, differentiate required amount of times.
    \item express known terms in power series and do operations on them instead
    \item compare coefficients, given that an exact value is known
\end{enumerate}
\\Some of the common Maclaurin series:\[
    \frac{1}{1-x} = \sum_{n=0}^{\infty} x^{n} = 1 + x + x^{2} + x^{3} + \cdots 
\]\[
    e^{x} = \sum_{n=0}^{\infty} \frac{x^{n}}{n!} = 1 + x + \frac{x^{2}}{2!} + \frac{x^{3}}{3!} + \cdots
\]\[
\cos{x} = \sum_{n=0}^{\infty} {(-1)}^{n} \frac{x^{2n}}{(2n)!}
= 1 - \frac{x^{2}}{2!} + \frac{x^{4}}{4!} - \frac{x^{6}}{6!} + \cdots
\]\[
\sin{x} = \sum_{n=0}^{\infty} {(-1)}^{n}\frac{x^{2n+1}}{(2n+1)!}
= x - \frac{x^{3}}{3!} + \frac{x^{5}}{5!} - \frac{x^{7}}{7!} + \cdots
\]\[
\tan^{-1}{x} = \sum_{n=0}^{\infty} {(-1)}^{n} \frac{x^{2n+1}}{2n+1}
= x - \frac{x^{3}}{3} + \frac{x^{5}}{5} - \frac{x^{7}}{7} + \cdots
\]\[
\log{(1+x)} = \sum_{n=1}^{\infty} {(-1)}^{n-1} \frac{x^{n}}{n} 
= x - \frac{x^{2}}{2} + \frac{x^{3}}{3} - \frac{x^{4}}{4} + \cdots
\]\[
\log{(1-x)} = \sum_{n=1}^{\infty} -\frac{x^{n}}{n}
= - x - \frac{x^{2}}{2} - \frac{x^{3}}{3} - \frac{x^{4}}{4} - \cdots
\]
\newmdtheoremenv[style=defEnv]{Taylor's theorem}[theorem]{(Taylor Series) Theorem}
\begin{Taylor's theorem}
    Let $f$ be a function defined on a closed interval between two numbers $x_0$ and $x$.
    Assume that the function has $n+1$ derivatives on the interval and that they are all continuous. Then \[
        f(x) = \sum_{k=0}^{n} \frac{f^{(k)}(x_0)}{k!} {(x-x_0)}^{k} + R_n
    \]where the \emph{remainder} $R_n$ is given by\[
    R_n = \int_{x_0}^{x} \frac{{(x-t)}^{n}}{n!} f^{(n+1)}(t) \mathrm{d}t
    \]
\end{Taylor's theorem}
\begin{proof}
    Use integration by parts. From the FTC, we have \[
        f(x) = f(x_0) + \int_{x_0}^{x} f'(t)\mathrm{d}t 
    \]Now $\int_{x_0}^{x} f'(t)\mathrm{d}t = \int_{x_0}^{x} f'(t)\mathrm{d}(-(x-t))$, and use integration by parts,
    \begin{eqnarray*}
        \int_{x_0}^{x} f'(t)\mathrm{d}t
        &=& +(t-x)f'(t)\big|^x_{x_0} - \int_{x_0}^{x} (t-x)f''(t)\mathrm{d}t \\
        &=& (x-x_0)f'(x_0) + \int_{x_0}^{x} (x-t)f^{(2)}(t)\mathrm{d}t \\
        &=& (x-x_0)f'(x_0) + \frac{{(x-x_0)}^{2}}{2}f^{(2)}(x_0) 
        + \int_{x_0}^{x} \frac{{(x-t)}^{2}}{2}f^{(3)}(t)\mathrm{d}t \\
        &\vdots&
    \end{eqnarray*}
    Repeat $n$ tiems to get the result.
\end{proof}
\subsection{Estimation Accuracy}
Alternative form of the remainder:\[
    R_n = \int_{x_0}^{x} \frac{{(x-t)}^{n}}{n!} f^{(n+1)}(t)\mathrm{d}t
\]Use the generalized MVT for integrals, since $x_0 \le t \le x$, $x-t \ge 0$, $g(t) := \frac{{(x-t)}^{n}}{n!} \ge 0$.
Thus $\exists \,c \in [x_0, x]$ s.t. \[
    R_n = f^{(n+1)}(c) \int_{x_0}^{x} \frac{{(x-t)}^{n}}{n!} \mathrm{d}t
\]\[
\Rightarrow R_n = \frac{f^{(n+1)}(c)}{(n+1)!} {(x-x_0)}^{n+1}
\]
\fbox{\begin{minipage}{\linewidth}
\emph{Converges to $f(x)$ if $R_n \rightarrow 0$ as $n \rightarrow \infty$!}
\end{minipage}}
\medskip
\\Here is a very useful alternative of Taylor's theorem that we use in Numerical Analysis.
Put $x = x_0 + h$ (and after that $x_0 \rightarrow x$ if you want)\[
    f(x_0 + h) = \sum_{k=0}^{n} \frac{h^{k}}{k!}f^{(k)}(x_0) + R_n(x_0,h)
\]
\begin{eqnarray*}
    R_n &=& \int_{x_0}^{x_0 + h} \frac{{(x_0 + h - t)}^{n}}{n!} f^{(n+1)}(t)\mathrm{d}t \\
        &=& \frac{h^{n+1}}{(n+1)!} f^{(n+1)}(c) \quad \text{, where $c \in [x_0, x_0 + h]$}
\end{eqnarray*}
In terms of \emph{bounding the remainder}, for maclaurin series with $x_0 = 0$, for instance, use form\[
    R_n = \frac{f^{(n+1)}(c)}{(n+1)!} x^{n+1}
\]If $\left|f^{(n+1)}(c)\right| \le M_{n+1}$, then \[
|R_n| \le M_{n+1} \frac{{|x|}^{n+1}}{(n+1)!}
\]
\begin{ex}
    \,

    \begin{enumerate}[label = (\arabic*)]
        \item \[
            f(x) = \sin{x} = \sum_{k=0}^{n} {(-1)}^{k}\frac{x^{2k+1}}{(2k+1)!} + R_{2n+3}
        \]where\[
            |R_{2n+3}| = \left|\frac{f^{(2n+3)}(x)}{(2n+3)!}x^{2n+3}\right| \le \frac{{|x|}^{2n+3}}{(2n+3)!}
        \]Hence \[
        \sin{(0.1)} \approx 0.1 - \frac{10^{-3}}{3!} + \frac{10^{-5}}{5!}
        \]with an error which is less than\[
            \frac{0.1^{7}}{7!} = \frac{10^{-7}}{5040} < 10^{-10}
        \]
        \item Compute $\sin{(\frac{\pi}{6} + 0.2)}$ to an accuracy of $10^{-4}$.
            Even though the power series of $\sin{x}$ will converge if we take enough terms,
            since $\frac{\pi}{6} + 0.2$ is not small, we will need \emph{a lot of terms} to get to the required accuracy.
            \\It is \emph{much} better to expand about $\frac{\pi}{6}$ using the formula \[
                f(x_0 + h) = \sum_{k=0}^{n} \frac{h^{k}}{k!}f^{(k)}(x_0) + R_n(x_0,h)
            \]Now $h = 0.2$, $f^{(n+1)}$ is $\sin$ or $\cos$, \[
            \Rightarrow |R_n| \le \frac{0.2^{n+1}}{(n+1)!}
            \Rightarrow R_3 \le \frac{0.2^{4}}{24} = \frac{16 \times 10^{-4}}{24} < 10^{-4}
            \]\[
            \Rightarrow \sin{(\frac{\pi}{6})} \approx \sin{\frac{\pi}{6}} + 0.2\cos{\frac{\pi}{6}} 
            + \frac{0.2^{2}}{2!} \left(-\sin{\frac{\pi}{6}}\right) + \frac{0.2^{3}}{3!}\left(-\cos{\frac{\pi}{6}}\right) 
            \]
        \item Compute $e$ to 3 decimals. \[
                e^{x} = 1 + x + \frac{x^{2}}{2!} + \cdots + \frac{x^{n}}{n!} + \frac{e^{c}}{(n+1)!} x^{n+1}
            \]If $x<0$ then $c<0$ and $|R_n| \le \frac{{|x|}^{n+1}}{(n+1)!}$;
            if $x>0$ and such that $x \le b$ say, then $|R_n| \le \frac{e^{b}}{(n+1)!} b^{n+1}$

        Showed earlier (see chapter on logarithms) that $2<e<4$. Therefore\[
                |R_n| \le \frac{e}{(n+1)!} \le \frac{4}{(n+1)!}
            \]Need $\frac{4}{(n+1)!}$ to be less than $10^{-3}$. \[
                |R_5| = \frac{4}{6!} = \frac{1}{180} > 10^{-3}
            \]\[
                |R_6| = \frac{4}{7!} = \frac{1}{1260} < 10^{-3}
            \]So \[
                e \approx 1 + 1 + \frac{1}{2!} + \frac{1}{3!} + \frac{1}{4!} + \frac{1}{5!} + \frac{1}{6!} + R_6
            \]
        \item Compute $\log{1.1}$ to 3 decimals.\[
                |R_n| = \frac{f^{n+1}(0.1)}{(n+1)!} 0.1^{n+1} < \frac{0.1^{n+1}}{(n+1)!}
        \]\[
            |R_1| < \frac{0.1^{2}}{2!} > 10^{-3}
        \]\[
            |R_2| < \frac{0.1^{3}}{3!} < 10^{-3}
        \]So\[
        \log{1.1} \approx 0.1 - \frac{0.1^{2}}{2} + R_2
        \]
    \end{enumerate}
\end{ex}

\subsection{Alternative to L'Hôpital's rule}
Sometimes it is easier to use Taylor's theorem instead of differentiating many times.
\begin{ex}
    \,

    \begin{enumerate}[label = (\arabic*)]
        \item 
            \begin{eqnarray*}
                \lim_{x \rightarrow 0} \frac{\sin{x} - x}{\tan{x} - x}
                &=& \lim_{x \rightarrow 0} \frac{\sin{x}\cos{x} - x\cos{x}}{\sin{x} - x\cos{x}} \\
                &=& \lim_{x \rightarrow 0} \frac{\left(x - \frac{x^{3}}{6} + \cdots\right) 
                \left(1-\frac{x^{2}}{2} + \cdots\right) - x \left(1 - \frac{x^{2}}{2} + \cdots\right)}
                {x - \frac{x^{3}}{6} + \cdots - x\left(1 - \frac{x^{2}}{2} + \cdots\right) } \\
                &=& \lim_{x \rightarrow 0} \frac{-\frac{x^{3}}{6} + \cdots}{\frac{x^{3}}{3}} \\
                &=& -\frac{1}{2}
            \end{eqnarray*}
        \item 
            \begin{eqnarray*}
                \lim_{x \rightarrow 1} \frac{\log{x}}{e^{x} - e}
                &=& \lim_{y \rightarrow 0} \frac{\log{(1+y)}}{e(e^{y}-1)} \qquad \text{(put $x = 1 + y$)}\\
                &=& \lim_{y \rightarrow 0} \frac{y - \frac{y^{2}}{2} + \frac{y^{3}}{3} + \cdots}
                {e(1 + y + \cdots - 1)} \\
                &=& \frac{1}{e}
            \end{eqnarray*}
    \end{enumerate}
\end{ex}

\chapter{Trigonometric Series --- Fourier Series}
\section{Orthogonal and orthonormal function spaces}
\newmdtheoremenv[style=defEnv]{inner product of f and g}[theorem]{Definition}
\begin{inner product of f and g}
    If $f$, $g$ are real valuedtfunctions that are Riemann integrable on $[a,b]$, 
    then we define the inner product of $f$ and $g$, denoted by $(f,g)$, by \[
        (f,g) := \int_{a}^{b} f(x)g(x) \mathrm{d}x
    \]\underline{Note} ${(f,f)}^{\frac{1}{2}} = {\left(\int_{a}^{b} f^{2} \mathrm{d}x\right)}^{\frac{1}{2}} 
    := \parallel f \parallel \ge 0$
\end{inner product of f and g}
\newmdtheoremenv[style=defEnv]{orthogonal system}[theorem]{Definition}
\begin{orthogonal system}
    Let $S = \{\phi_0, \phi_1, \phi_2, \ldots\}$ be a collection of functions that are Riemann integrable on $[a,b]$.
    If $(\phi_n, \phi_m) = 0$ whenever $m \neq n$, then $S$ is an \emph{orthogonal} system on $[a,b]$.
    
    If in addition $\left\|\phi_n\right\| = 1$, i.e. $\int_{a}^{b} \phi_n^{2} \mathrm{d}x = 1$,
    then $S$ is said to be \emph{orthonormal} on $[a,b]$.

    \underline{Note}: Can easiily go from orthogonal to orthonomal by considering $\frac{\phi_n}{\left\|\phi_n\right\| }$
\end{orthogonal system}
The orthonormal \emph{trigonometric system} will be used, where \[
    \phi_0(x) = \frac{1}{\sqrt{2\pi}}, \phi_{2n-1}(x) = \frac{\cos{(nx)}}{\sqrt{\pi}},
    \phi_{2n}(x) = \frac{\sin{(nx)}}{\sqrt{\pi}} \qquad (n = 1,2,\ldots)
\]i.e. The system is\[
S = \left\{\frac{1}{\sqrt{2\pi}}, \frac{\cos{x}}{\sqrt{\pi}}, \frac{\sin{x}}{\sqrt{\pi}},
\frac{\cos{(2x)}}{\sqrt{\pi}}, \frac{\sin{(2x)}}{\sqrt{\pi}}, \ldots\right\}
\]
$S$ defined above is orthonormal on any interval of length $2\pi$, e.g. $[0,2\pi]$, $[-\pi,\pi]$, etc.


\section{Periodic functions and periodic extensions}
A function $f(x)$ is periodic with period $T$ if $\forall x$,\[
    f(x + T) = f(x)
\]It follows that a $T$-periodic function is also $mT$-periodic $\forall m \in \mathbb{Z}$, i.e.\[
f(x \pm mT) = f(x)
\]
Start with any continuous function $f(x)$ in an interval $a \le x < b$. 
Can extend this periodically to have period $T = b - a$.
Function on $[a,b)$ extended periodically and the new function is discontinuous at %chktex 9
$x = a + mT \;\forall m \in \mathbb{Z}$.

\newmdtheoremenv[style=defEnv]{fn at points of discontinuity}[theorem]{Definition}
\begin{fn at points of discontinuity}
    At points of discontinuity $x = \xi$, define\[
        f(\xi) = \frac{1}{2} \left[f(\xi_+) + f(\xi_-)\right] 
    \]
\end{fn at points of discontinuity}

\subsection{Integrals over a period}
For a periodic function $f(x)$ of period $T$ and for arbitrary values of $a$, we have \[
    \int_{-a}^{T-a} f(x)\mathrm{d}x = \int_{0}^{T} f(x)\mathrm{d}x
\]
In fact, \[
    \int_{\alpha}^{\beta} f(x)\mathrm{d}x = \int_{\alpha + T}^{\beta + T} f(x)\mathrm{d}x
\]
\begin{proof}
    \[
        \int_{\alpha}^{\beta} f(x)\mathrm{d}x = \int_{\alpha + T}^{\beta + T} f(y-T)\mathrm{d}y 
        = \int_{\alpha + T}^{\beta + T} f(y)\mathrm{d}y = \int_{\alpha + T}^{\beta + T} f(x)\mathrm{d}x
    \]
\end{proof}

\section{Superposition of harmonics and trigonometric polynomials}
Start with an oscillation $\sin{(\omega x)}$. Here $\omega$ is the frequency. The period is $T = \frac{2\pi}{\omega}$.

This is a ``pure'' harmonic oscillation. Signals --- e.g.\ sound, electromagnetic waves, water waves, 
are not pure oscillations, they contain \emph{higher harmonics}.

Lets add another oscillation of frequency $2\omega$, i.e. $\sin{(2\omega x)}$, whose period is 
$T_2 = \frac{2\pi}{2\omega} = \frac{\pi}{\omega}$. This is called the \emph{1st harmonic}.

Signal could be \[
    S_2(x) = A_1\sin{(\omega x)} + B_2\sin{(2\omega x)}
\]
$S_2(x)$ has period $T = \frac{2\pi}{\omega}$ overall. 1st harmonic has period $\frac{T}{2}$.

Can add more and more higher frequencies and in fact can produce a wave (oscillation) that is a 
\emph{trigonometric polynomial} defined by\[
    S_n(x) = \frac{1}{2} a_0 + \sum_{k=1}^{n} [a_k\cos{(k\omega x)} + b_k\sin{(k\omega x)}]
\]
The constant $\frac{1}{2} a_0$ is included. ($\frac{1}{2}$ is useful as we will see later)

\underline{Note}: Went from $\omega_1 = \omega$ to $\omega_2 = 2\omega$ etc. All the frequencies have ratios that are rational.
If $\frac{\omega_1}{\omega_2}$ is irrational, we get \emph{quasi-periodic} oscillations.

\section{Complex Notation}
Useful to use Euler's relation\[
    \cos{\theta} + i\sin{\theta} = e^{i\theta}
\] and since $\cos{\theta} - i\sin{\theta} = e^{-i\theta}$, \[
\cos{\theta} = \frac{1}{2} \left(e^{i\theta} + e^{-i\theta}\right), \quad 
\sin{\theta} = \frac{1}{2i} \left(e^{i\theta} - e^{-i\theta}\right) 
\]
So, can represent everything as complex and find real expressions by taking real or imaginary parts.\ e.g.\[
    \frac{\mathrm{d}}{\mathrm{d}x} ae^{i\omega(x - \phi)} = ai\omega e^{i\omega(x - \phi)}
\]
Can also integrate \[
    \int e^{inx}\mathrm{d}x = \int (\cos{nx} + i\sin{nx})\mathrm{d}x 
    = \left[\frac{\sin{nx}}{n} - i \frac{\cos{nx}}{n}\right] = \frac{e^{inx}}{in}
\]

\subsection*{Orthogonality}
\[
    \int_{-\pi}^{\pi} e^{inx}\mathrm{d}x = 
    \begin{cases}
        0 & n \neq 0 \\
        2\pi & n = 0
    \end{cases}
\]
For any integers $m$, $n$ we have\[
    \int_{-\pi}^{\pi} e^{inx}e^{-imx}\mathrm{d}x =
    \begin{cases}
        0 & n \neq m \\
        2\pi & n = m
    \end{cases}
\]
\subsection{Complex notation for trigonometric polynomials}
Start with the polynomial (have set $\omega = 1$)\[
    S_n(x) = \frac{1}{2}a_0 + \sum_{k=1}^{n} (a_k \cos{kx} + b_k \sin{kx})
\] Use the relations found earlier
\begin{eqnarray*}
    S_n(x) &=& \frac{1}{2}a_0 + \sum_{k=1}^{n} a_k \left(\frac{e^{ikx} + e^{-ikx}}{2}\right) 
            + b_k \left(\frac{e^{ikx} - e^{-ikx}}{2i}\right) \\
           &=& \frac{1}{2}a_0 + \sum_{k=1}^{n} \frac{1}{2}(a_k - ib_k)e^{ikx} + 
           \sum_{k=1}^{n} \frac{1}{2}(a_k + ib_k)e^{-ikx}
\end{eqnarray*}
Can now write this as a single complex series as follows
\begin{equation}
    S_n(x) = \sum_{k = -n}^{n} r_k e^{ikx} \label{eq:1}
\end{equation}
where\[
\left.
    \begin{align*}
        & r_0 = \frac{1}{2} a_0 \\
        & r_k = \frac{1}{2} (a_k - ib_k) \\
        & r_{-k} = \frac{1}{2} (a_k + ib_k)
    \end{align*}
\quad\right\} k = 1,2,\ldots,n
\]
Notice that $r_k = r_{-k}^*$, where $*$ denotes complex conjugate.
This is \emph{not accidental}. $S_n(x) \in \mathbb{R}$ in equation\eqref{eq:1}.
Hence it must be equal to its complex conjugate.\[
    S_n(x) = \sum_{k=-n}^{n} r_k e^{ikx}, \, S_n^*(x) = \sum_{k=-n}^{n} r_k^{*}e^{-ikx}
\]Change indexing, put $k = -l$, to find\[
S_n^*(x) = \sum_{l=n}^{-n} r_{-l}^{*} e^{ilx} = \sum_{l=-n}^{n} r_{-l}^{*}e^{ilx}
= \sum_{k=-n}^{n} r_{-k}^{*} e^{ikx}
\]
Comparing the two, we can see that they are equal iff \[
    r_k = r_{-k}^{*} \quad \text{i.e.} \quad r_k^{*} = r_{-k}
\]
Conversely, if we are given a complex form $f(x) = \sum_{k=-n}^{n} r_k e^{ikx}$,
then $f(x) \in \mathbb{R} \iff r_k = r_{-k}^{*}$, i.e. \[
    r_k + r_{-k} = r_k + r_k^{*} \quad \text{is real}
\]\[
r_k - r_{-k} = r_k - r_k^{*} \quad \text{is imaginary}
\]
\begin{ex}
    Take $S_n(x) = \cos{x} + \frac{1}{2} \sin{x} + 3\cos{2x}$. Express as a complex trigonometric series.
    \begin{eqnarray*}
        S_n &=& \frac{1}{2}(e^{ix} + e^{-ix}) - \frac{i}{4} (e^{ix} - e^{-ix}) + \frac{3}{2} (e^{2ix} + e^{-2ix}) \\
            &=& \left(\frac{1}{2} - \frac{i}{4}\right) e^{ix} + \left(\frac{1}{2} + \frac{i}{4}\right) e^{-ix}
            + \frac{3}{2}e^{2ix} + \frac{3}{2}e^{-2ix} \\
            &=& \sum_{k=-2}^{2} r_k e^{ikx}
    \end{eqnarray*}
    where $r_0 = 0, r_1 = \frac{1}{2} - \frac{i}{4}, r_{-1} = \frac{1}{2} + \frac{i}{4} = r_1^{*},
    r_2 = \frac{3}{2}, r_{-2} = \frac{3}{2} = r_2^{*}$.
\end{ex}

\section{Fourier Series}
Consider the trigonometric polynomial
\begin{equation}
    f(x) = S_n(x) = \frac{1}{2}a_0 + \sum_{n=1}^{N} (a_n \cos{nx} + b_n \sin{nx}) \label{eq:2}
\end{equation}
There are $2N + 1$ coefficients to determine.
Use orthogonality on the interval $[-\pi, \pi]$ for $\sin{mx}, \cos{nx}$, etc.\[
    \int_{-\pi}^{\pi} \sin{mx} \cos{nx} \mathrm{d}x 
    = \int_{-\pi}^{\pi} \frac{1}{2} [\sin{(m+n)x} + \sin{(m-n)x}]\mathrm{d}x = 0
\]\[
\int_{-\pi}^{\pi} \sin{mx} \sin{nx}\mathrm{d}x
= \int_{-\pi}^{\pi} \frac{1}{2} [\cos{(m-n)x} - \cos{(m+n)x}] \mathrm{d}x = 
\begin{cases}
    0 & \text{if } m \neq n \\
    \pi & \text{if } m = n
\end{cases}
\]\[
\int_{-\pi}^{\pi} \cos{mx} \cos{nx}\mathrm{d}x
= \int_{-\pi}^{\pi} \frac{1}{2} [\cos{(m-n)x} + \cos{(m+n)x}] \mathrm{d}x = 
\begin{cases}
    0 & \text{if } m \neq n \\
    \pi & \text{if } m = n
\end{cases}
\]
The constant $a_0$ can be found immediately by integrating over $[-\pi, \pi]$,\[
    \int_{-\pi}^{\pi} f(x)\mathrm{d}x 
    = 2\pi a_0 + \sum_{k=1}^{N} \int_{-\pi}^{\pi} (a_k \cos{kx} + b_k \sin{kx}) \mathrm{d}x = 2\pi a_0
\]\[
\Rightarrow a_0 = \frac{1}{2\pi}\int_{-\pi}^{\pi} f(x)\mathrm{d}x
\]
i.e. $a_0$ is the average (or mean) value of the function over the domain.
\smallskip
\\Next, take any integer $m \ge 1$, multiple $\eqref{eq:2}$ by $\cos{mx}$ and integrate over $[-\pi, \pi]$. 
Using the \emph{orthogonality} property, we see that only the term containing $a_m$ in the sum will survive to give\[
    \int_{-\pi}^{\pi} f(x) \cos{mx}\mathrm{d}x = a_m \int_{-\pi}^{\pi} \cos^{2}{mx}\mathrm{d}x = \pi a_m
\]\[
    \Rightarrow a_m = \frac{1}{\pi} \int_{-\pi}^{\pi} f(x)\cos{mx}\mathrm{d}x
\]
Similarly, \[
    b_m = \frac{1}{\pi} \int_{-\pi}^{\pi} f(x)\sin{mx}\mathrm{d}x
\]

\newmdtheoremenv[style=defEnv]{fourier series}[theorem]{(Fourier Series) Theorem}
\begin{fourier series}%
\label{thm:fourierSeriesTheorem}
    The fourier series\[
        \frac{1}{2}a_0 + \sum_{n=1}^{\infty} (a_n \cos{nx} + b_n \sin{nx})
    \]or\[
        \sum_{n=-\infty}^{\infty} r_n e^{inx}
    \] formed by the fourier coefficients \[
    a_n = \frac{1}{\pi}\int_{-\pi}^{\pi} f(x)\cos{nx}\mathrm{d}x \quad \text{\&} \quad
    b_n = \frac{1}{\pi}\int_{-\pi}^{\pi} f(x)\sin{nx}\mathrm{d}x
    \]
    converges to the value $f(x)$ for any piecewise continuous $f(x)$ of period $2\pi$ 
    which has piecewise continuous derivatives of first and second order.*
    \\At any discontinuities, the value of the function must be defined by $f(x) = \frac{1}{2}[f(x_+) + f(x_-)]$.
\end{fourier series}
*\underline{Note}: Can relax the assumption of the second derivative. 
It is enough to have $f'(x)$ be piecewise continuous, i.e.\ the functon is pieceswise smooth.
If $f(x)$ is continuous, the convergence is \emph{absolute} and \emph{uniform}. 
If it is discontinuous, absolute and uniform convergence everywhere \emph{except at the discontinuity}.
\medskip
\\For the proof we will need some additional lemmas.
\subsection{A trigonometric formula}
We will prove the following --- needed later
\begin{eqnarray*}
    C_n(x) &=& \frac{1}{2} + \cos{x} + \cos{2x} + \cdots + \cos{nx} \\
           &=& \frac{\sin{(n+1)x}}{2\sin{\frac{1}{2}x}}
\end{eqnarray*}
Clearly $\frac{1}{2}x \neq 0,\pm\pi,\pm 2\pi,\ldots$ i.e. $x \neq 0,\pm 2\pi, \pm 4\pi, \ldots$
\smallskip
\\
\begin{minipage}{\linewidth}
If we definie $C_n(x)$ at these points by $(n+\frac{1}{2})$, then the function is continuous everywhere.
\end{minipage}
\[
    \lim_{x \rightarrow 2k\pi} \frac{\sin{(n+\frac{1}{2})x}}{2\sin{\frac{1}{2}x}}
    = \frac{\left(n+\frac{1}{2}\right) \cos{(n+\frac{1}{2})x}}{\cos{\frac{1}{2}x}}
    = n + \frac{1}{2}
\]
Use $\cos{kx} = \frac{1}{2} (e^{ikx} + e^{-ikx})$ to re-write\[
    C_n(x) = \frac{1}{2} \sum_{k=-n}^{n} e^{ikx} 
    = \frac{1}{2}(e^{-inx} + e^{ix}e^{-inx} + {\left(e^{ix}\right)}^{2}e^{-inx} + \cdots + e^{inx})
\]
i.e.\ a geometric progression with ratio $r = e^{ix} = \cos{x} + i\sin{x}$.
\smallskip
\\Now $r = 1$ only if $x = 0,\pm 2\pi, \ldots$, i.e.\ the exceptional points that we excluded (treated separatedly).
Sum it up to find \[
    C_n(x) = \frac{1}{2}e^{-inx}\frac{1 - r^{2n+1}}{1 - r}
    = \frac{1}{2} \left[e^{-inx} - e^{i(n+1)x}\right] \frac{1}{1 - e^{ix}}
\]
Multiply top + bottom by $e^{-\frac{1}{2}ix}$
\begin{eqnarray*}
    \Rightarrow C_n(x) &=& \frac{1}{2}
    \frac{[e^{-i(n+\frac{1}{2})x} - e^{i(n+\frac{1}{2})x}]}{e^{-\frac{1}{2}ix} - e^{\frac{1}{2}ix}} \\
                       &=& \frac{\sin{(n+\frac{1}{2})x}}{2\sin{\frac{1}{2}x}}
\end{eqnarray*}
Integrate from 0 to $\pi$ we find\[
    \int_{0}^{\pi} \frac{\sin(n+\frac{1}{2})t}{2\sin{\frac{1}{2}t}} \mathrm{d}t
    = \int_{0}^{\pi} \left(\frac{1}{2} + \sum_{k=1}^{n} \cos{kt}\right) \mathrm{d}t
    = \frac{\pi}{2}
\]
\subsection{Lemmas}
\newmdtheoremenv[style=defEnv]{Riemann-lebesgue}[theorem]{(Riemann-Lebesgue) Lemma}
\begin{Riemann-lebesgue}
    If the function $g(x)$ is integrable on $[a,b]$ (e.g.\ it is piecewise continuous), then\[
        I_\lambda = \int_{a}^{b} g(x) \sin{\lambda x}\,\mathrm{d}x
    \] tends to 0 as $\lambda \rightarrow \infty$
\end{Riemann-lebesgue}
\begin{proof}
    (Will do it when $g'$ is also piecewise continuous. For the general case see HW problems.)
    \\Using integration by parts,
    \begin{eqnarray*}
        I_\lambda &=& \int_{a}^{b} g(x)\sin{\lambda x} \,\mathrm{d}x \\
                  &=& {\left[-\frac{\cos{\lambda x}}{\lambda}g(x)\right]}_a^{b}
                  + \int_{a}^{b} \frac{\cos{\lambda x}}{\lambda}g'(x) \mathrm{d}x \\
                  &=& \frac{1}{\lambda}\left[g(a)\cos{\lambda a} - g(b)\cos{\lambda b} 
                  + \int_{a}^{b} \cos{\lambda x} g'(x) \mathrm{d}x\right] 
    \end{eqnarray*}
    \[
        \Rightarrow |I_\lambda | \le \frac{1}{\lambda}M \quad 
        \text{for some constant M}
    \]
    And the result follows.
\end{proof}
\newmdtheoremenv[style=defEnv]{Lemma 2}[theorem]{Lemma}
\begin{Lemma 2}
    \[
        \int_{0}^{\infty} \frac{\sin{z}}{z}\mathrm{d}z = \frac{\pi}{2}
    \]
\end{Lemma 2}
\begin{proof}
    Show improper integral exists, i.e. \[
        \exists I = \lim_{M \rightarrow \infty} \int_{0}^{M} \frac{\sin{z}}{z}\mathrm{d}z
    \] 
    (\underline{Note} $z = 0$ is not a problem.)
    \\Consider $0 < M < N$ and calculate
    \begin{eqnarray*}
        I_N - I_M &=& \int_{M}^{N} \frac{\sin{z}}{z}\mathrm{d}z \\
                  &=& -\left.\frac{\cos{z}}{z}\right|_M^{N} - \int_{M}^{N} \frac{\cos{z}}{z^{2}}\mathrm{d}z \\
                  &=& \frac{\cos{M}}{M} - \frac{\cos{N}}{N} - \int_{M}^{N} \frac{\cos{z}}{z^{2}}\mathrm{d}z \\
                  &\le& \frac{1}{M} + \frac{1}{N} + \int_{M}^{N} \frac{\mathrm{d}z}{z^{2}} \\
                  &=& \frac{2}{M}
    \end{eqnarray*}
    Hence convergence since $|I_N - I_M|$ can be made arbitrarily small (Cauchy).
    In fact letting $N \rightarrow \infty$, we see that $|I - I_M| \le \frac{2}{M}$,
    so $I_M$ approaches its limit algebraically.
    \\Now take $p > 0$ arbitrarily and pick $M = \lambda p$.\[
        I_M = I_{\lambda p} = \int_{0}^{\lambda p} \frac{\sin{z}}{z}\mathrm{d}z
        = \int_{0}^{p} \frac{\sin{\lambda x}}{\lambda x} \lambda\mathrm{d}x
        = \int_{0}^{p} \frac{\sin{\lambda x}}{x}\mathrm{d}x
    \]
    where we have now \emph{fixed} the integration range to $[0,p]$.
    As $M \rightarrow \infty$, $\lambda p \rightarrow \infty$, i.e. $\lambda \rightarrow \infty$,
    and by the estimate above \[
        \left|I - \int_{0}^{p} \frac{\sin{\lambda x}}{x}\mathrm{d}x\right| \le \frac{2}{M} = \frac{2}{\lambda p}
    \]
    i.e. 
    \begin{equation}
        \lim_{\lambda \rightarrow \infty} \int_{0}^{p} \frac{\sin{\lambda x}}{x}\mathrm{d}x = I \quad 
        \forall p \text{\ that are sufficiently big} \label{eq:3}
    \end{equation}
    
    Cannot apply Riemann-Lebesgue directly. Consider the function \[
        h(x) =
        \begin{cases}
            \frac{1}{x} - \frac{1}{2\sin{\frac{x}{2}}} & x \neq 0 \\
            0 & x=0
        \end{cases}
    \]
    \underline{Fact}: $h(x)$ is continuous and also has a continuous first derivative for $0 \le x < \pi$,
    since by using taylor series expansion, 
    $\frac{1}{x} - \frac{1}{2\sin{\frac{x}{2}}} = -\frac{x}{24} + Kx^{3} + \cdots$.
    \smallskip
    \\Now use the Riemann-Lebesgue Lemma to see that for $0 \le p < 2\pi$,\[
        \int_{0}^{p} \sin{\lambda x} \left(\frac{1}{x} - \frac{1}{2\sin{\frac{x}{2}}}\right) \mathrm{d}x \rightarrow 0
        \quad \text{as} \quad \lambda \rightarrow \infty
    \]
    \underline{Note}: The convergence is uniform for $0 \le p \le \pi$ since $|h(x)|$ and $|h'(x)|$ 
    are both bounded in this interval.
    \smallskip
    \\From~\eqref{eq:3} we have immediately \[
        \lim_{\lambda \rightarrow \infty}\int_{0}^{p} \frac{\sin{\lambda x}}{2\sin{\frac{x}{2}}} \mathrm{d}x = I
    \]
    Pick $\lambda = n + \frac{1}{2}$ and $p = \pi$, we have shown already that 
    $\int_{0}^{\pi} \frac{\sin{(n+\frac{1}{2})x}}{2\sin{\frac{x}{2}}} \mathrm{d}x = \frac{\pi}{2}$
    independent of $n$.
    Hence we have proved:\[
        I = \int_{0}^{\infty} \frac{\sin{z}}{z}\mathrm{d}z = \frac{\pi}{2}
    \]
\end{proof}

\subsection{Proof of Fourier Series}
Start with $n$th ``Fourier polynomial''\[
    S_n(x) = \frac{1}{2}a_0 + \sum_{k=1}^{n} (a_k \cos{kx} + b_k \sin{kx})
\]
and substitute the formulas for $a_k$ $b_k$, change order of summation and integration (finite sum, so ok), to find
\begin{eqnarray*}
    S_n(x) &=& \frac{1}{\pi} \int_{-\pi}^{\pi} f(t)\left[\frac{1}{2} + 
           \sum_{k=1}^{n} (\cos{kt}\cos{kx} + \sin{kt}\sin{kx})\right] \mathrm{d}t \\
           &=& \frac{1}{\pi}\int_{-\pi}^{\pi} f(t) \left[\frac{1}{2} 
           + \sum_{k=1}^{n} \cos{k(t-x)}\right] \mathrm{d}t \\
           &=& \frac{1}{\pi}\int_{-\pi}^{\pi} f(t) 
           \frac{\sin{\left[\left(n + \frac{1}{2}\right) \left(t-x\right) \right]}}
           {2\sin{\frac{1}{2}(t-x)}} \mathrm{d}t \\
           &=& \frac{1}{\pi}\int_{-\pi-x}^{\pi-x} 
           \frac{f(x + \xi) \sin{(n + \frac{1}{2})\xi}}{2\sin{\frac{1}{2}\xi}} \mathrm{d}\xi \qquad 
           \text{(substitute $\xi = t - x$)} \\
           &=& \frac{1}{\pi} \int_{-\pi}^{\pi} f(x + \xi)
           \frac{\sin{(n + \frac{1}{2})\xi}}{2\sin{\frac{1}{2}\xi}} \mathrm{d}\xi
\end{eqnarray*}
by using properties of integrals of periodic functions discussed earlier. Note that $x$ is a fixed number.
\medskip
\\We will prove that (and this proves Theorem~\ref{thm:fourierSeriesTheorem})\[
    \lim_{n \rightarrow \infty} \frac{1}{\pi}\int_{-\pi}^{\pi} f(x+\xi)
    \frac{\sin{(n + \frac{1}{2}) \xi}}{2\sin{\frac{1}{2} \xi}} \mathrm{d}\xi = f(x)
\]
At all points $x \in [-\pi, \pi]$, even points of discontinuity, we have \[
    f(x) = \frac{1}{2} [f(x_+) + f(x_-)]
\]
We have proven already that 
$\int_{0}^{\pi} \frac{\sin{(n+\frac{1}{2}) \xi}}{2\sin{\frac{1}{2} t}} \mathrm{d}t = \frac{\pi}{2}$,
and by a change of variable $t = -t'$ we also find 
$\int_{-\pi}^{0} \frac{\sin{(n + \frac{1}{2}) t'}}{2\sin{\frac{1}{2} t'}} \mathrm{d}t' = \frac{\pi}{2}$.
Hence \[
    f(x) = \frac{1}{\pi}\int_{0}^{\pi} f(x_+) \frac{\sin{(n+\frac{1}{2}) t}}{2\sin{\frac{1}{2} t}}\mathrm{d}t
    + \frac{1}{\pi}\int_{-\pi}^{0} f(x_-) \frac{\sin{(n+\frac{1}{2}) t}}{2\sin{\frac{1}{2} t}} \mathrm{d}t
\]
Using this identity gives\[
    \begin{split}
        S_n(x) - f(x) = \frac{1}{\pi}\int_{0}^{\pi} [f(x+\xi) - f(x_+)] 
        \frac{\sin{(n+\frac{1}{2})\xi}}{2\sin{\frac{1}{2}\xi}} \mathrm{d}\xi \\
        + \frac{1}{\pi}\int_{-\pi}^{0} [f(x+\xi) - f(x_-)]
        \frac{\sin{(n+\frac{1}{2})\xi}}{2\sin{\frac{1}{2}\xi}}\mathrm{d}\xi
    \end{split}
\]
What is left to do is to prove that 
\begin{enumerate}[label = (\roman*)]
    \item $\frac{f(x + \xi) - f(x_+)}{\sin{\frac{1}{2} \xi}}$ is piecewise continuous and so is its 1st derivative,
        on $0 \le \xi \le \pi$.
    \item $\frac{f(x+\xi) - f(x_-)}{\sin{\frac{1}{2} \xi}}$ is piecewise continuous along with its 1st derivative on
        $-\pi \le \xi \le 0$.
\end{enumerate}
Then by Riemann-Lebesgue Lemma $S_n \rightarrow f(x)$ as $n \rightarrow \infty$, i.e.\ convergence 
(uniform away from discontinuities).

\subsection{Examples of Fourier Series}
Will consider $f(x)$ to be $2\pi$-periodic.
\begin{enumerate}[label = (\roman*)]
    \item If $f(x)$ is even, i.e. $f(-x) = f(x)$, then $f(x)$ has only a \emph{cosine series}.\[
            a_n = \frac{2}{\pi}\int_{0}^{\pi} f(x) \cos{nx} \mathrm{d}x
    \]
    Similarly, if $f(x)$ is odd, $f(-x) = -f(x)$ then $f(x)$ has only a \emph{sine series}. \[
        b_n = \frac{2}{\pi} \int_{0}^{\pi} f(x) \cos{nx} \mathrm{d}x
    \]
\item If a function is defined on $[0,\pi]$ by an expression $f(x)$, then it can be extended as an \emph{even}
    or \emph{odd} function on $[-\pi,\pi]$.
\end{enumerate}

\subsection{Exponential complex form of Fourier Series}
Have already shown that for real $f(x)$,
\begin{eqnarray*}
    f(x) &=& \frac{1}{2} a_0 + \sum_{n=1}^{\infty} (a_n \cos{nx} + b_n \sin{nx}) \\
         &=& \sum_{n=-\infty}^{\infty} r_n e^{inx} \quad -\pi < x < \pi
\end{eqnarray*}
where\[
    \left. 
    \begin{align*}
        & r_n = \frac{1}{2} (a_n - ib_n) \\
        & r_{-n} = \frac{1}{2} (a_n + ib_n)
    \end{align*}
\right\} \text{for} \;n = 1,2,\ldots
\]
\begin{eqnarray*}
    r_n &=& \frac{1}{2} (a_n - ib_n) \\
        &=& \frac{1}{2} \frac{1}{\pi} \int_{-\pi}^{\pi} (f(x)\cos{nx} - if(x)\sin{nx}) \mathrm{d}x \\
        &=& \frac{1}{2\pi}\int_{-\pi}^{\pi} f(x) e^{-inx}\mathrm{d}x
\end{eqnarray*}
Similarly, \[
    r_n = \frac{1}{2\pi}\int_{-\pi}^{\pi} f(x) e^{inx}\mathrm{d}x
\]
(Clearly $r_n^{*} = r_{-n}$ since $f(x)$ is real)
\medskip
\\\fbox{\begin{minipage}{\linewidth}
Hence\[
    f(x) = \sum_{n=-\infty}^{\infty} r_n e^{inx}, -\pi < x < \pi
\] where \[
r_n = \frac{1}{2\pi}\int_{-\pi}^{\pi} f(x) e^{-inx}\mathrm{d}x, n = 0, \pm 1, \pm 2,\ldots
\]
\end{minipage}}
\subsection{Fourier Series over $2L$ periodic intervals}
The set of functions\[
    \frac{1}{\sqrt{2L}}, \frac{1}{\sqrt{L}}\cos{\left(n\frac{\pi x}{L}\right)}, 
    \frac{1}{\sqrt{L}}\sin{\left(n \frac{\pi x}{L}\right)}, \quad
    n = 1, 2, \ldots
\]are orthonormal on $[-L, L]$ (and in fact on any interval $[a, a+2L]$ since the function is periodic).
In addition,\[
    \int_{-L}^{L} \sin{n\frac{\pi x}{L}}\sin{m \frac{\pi x}{L}}\mathrm{d}x
        = \int_{-L}^{L} \cos{n \frac{\pi x}{L}} \cos{m \frac{\pi x}{L}}\mathrm{d}x = 
        \begin{cases}
            L & m = n \\
            0 & n \neq n
        \end{cases}
\]
and \[
    \int_{-L}^{L} \sin{n \frac{\pi x}{L}} \cos{m \frac{\pi x}{L}} \mathrm{d}x = 0
\]
\fbox{\begin{minipage}{\linewidth}
Therefore the real form is\[
f(x) = \frac{1}{2} a_0 + \sum_{n=1}^{\infty} \left[a_n \cos{n \frac{\pi x}{L}}
+ b_n \sin{n \frac{\pi x}{L}}\right] 
\]where\[
a_n = \frac{1}{L}\int_{-L}^{L} f(x) \cos{n \frac{\pi x}{L}}\mathrm{d}x \quad \& \quad 
b_n = \frac{1}{L}\int_{-L}^{L} f(x) \sin{n \frac{\pi x}{L}}\mathrm{d}x
\]
The complex form is\[
    f(x) = \sum_{-\infty}^{\infty} r_n e^{in\frac{\pi x}{L}}, |x| \le L
\]where\[
r_n = \frac{1}{2L} \int_{-L}^{L} f(x) e^{-in \frac{\pi x}{L}}\mathrm{d}x, n = 0, \pm 1, \pm 2, \ldots
\]
\end{minipage}}

\subsection{Parseval's Theorem}
\newmdtheoremenv[style=defEnv]{Parseval's Theorem}[theorem]{Theorem}
\begin{Parseval's Theorem}
    If $f(x)$ is represented by its Fourier series\[
        f(x) = \frac{1}{2} a_0 + \sum_{n=1}^{\infty} a_n \cos{nx} + b_n \sin{nx}, \;-\pi \le x \le \pi
    \]then we have \[
    \frac{1}{\pi}\int_{-\pi}^{\pi} {(f(x))}^{2} \mathrm{d}x 
    = \frac{1}{2}a_0^{2} + \sum_{n=1}^{\infty} (a^{2}_n + b^{2}_n)
    \]
\end{Parseval's Theorem}
\begin{proof}
    Easier to use complex notation \[
        f(x) = \sum_{n=-\infty}^{\infty} r_n e^{-inx}
    \]where\[
    r_n = \frac{1}{2}(a_n - ib_n), \,r_{-n} = \frac{1}{2} (a_n + ib_n) = r^{*}_n, \,r_0 = \frac{1}{2}a_0
    \]\[
    {(f(x))}^{2} = 
    \left(\sum_{n=-\infty}^{\infty} r_n e^{-inx}\right) 
    \left(\sum_{m=-\infty}^{\infty} r_m e^{-imx}\right) 
    \]Integrate and use orthogonality, \[
    \int_{-\pi}^{\pi} {(f(x))}^{2}\mathrm{d}x = 2\pi \sum_{n=-\infty}^{\infty} r_n r_{-n}
    = 2\pi\sum_{n=-\infty}^{\infty} {|r_n|}^{2}
    \]\[
    \Rightarrow \frac{1}{\pi} \int_{-\pi}^{\pi} {(f(x))}^{2}\mathrm{d}x 
    = \frac{1}{2} a_0^{2} + \sum_{n=1}^{\infty} (a_n^{2} + b_n^{2})
    \]
\end{proof}
\begin{ex}
    Compute the Fourier series of $\cos{\frac{x}{2}}$ over $(-\pi, \pi]$. %chktex 9
    Use Parseval's theorem to deduce the value of\[
        \sum_{n=1}^{\infty} \frac{1}{{(4n^{2} - 1)}^{2}}
    \]
    Function is even, so\[
        \cos{\frac{x}{2}} = \frac{1}{2} a_0 + \sum_{n=1}^{\infty} a_n \cos{nx}, \,-\pi \le x \le \pi
    \]\[
    a_0 = \frac{2}{\pi}\int_{0}^{\pi} \cos{\frac{x}{2}}\mathrm{d}x = \frac{4}{\pi}
    \]
    \begin{eqnarray*}
        a_n &=& \frac{2}{\pi} \int_{0}^{\pi} \cos{\frac{x}{2}} \cos{nx} \mathrm{d}x \\
            &=& \frac{2}{\pi} \int_{0}^{\pi} \frac{1}{2} 
            \left[\cos{\left(n+\frac{1}{2}\right)x} + \cos{\left(n-\frac{1}{2}\right)x}\right] \mathrm{d}x \\
            &=& \frac{1}{\pi} \left[\frac{\sin{(n+\frac{1}{2})\pi}}{n+\frac{1}{2}} 
            + \frac{\sin{(n-\frac{1}{2})\pi}}{n-\frac{1}{2}}\right] \\
            &=& \frac{1}{\pi}\left[\frac{\cos{n\pi}}{n+\frac{1}{2}} - \frac{\cos{n\pi}}{n-\frac{1}{2}}\right] \\
            &=& \frac{{(-1)}^{n}}{\pi}\left[\frac{2}{2n+1} - \frac{2}{2n-1}\right] \\
            &=& \frac{{(-1)}^{n}}{\pi} \frac{-4}{4n^{2}-1}
    \end{eqnarray*}
    By Parseval's theorem,
    \begin{eqnarray*}
        \frac{1}{\pi}\int_{-\pi}^{\pi} \cos^{2}{\frac{x}{2}} \mathrm{d}x
        &=& \frac{1}{2}a_0^{2} + \sum_{n=1}^{\infty} a_n^{2} \\
        &=& \frac{8}{\pi^{2}} + \frac{16}{\pi^{2}}\sum_{n=1}^{\infty} \frac{1}{{(4n^{2} - 1)}^{2}}
    \end{eqnarray*}
    $LHS = 1$, therefore\[
        \sum_{n=1}^{\infty} \frac{1}{{(4n^{2} - 1)}^{2}} = \frac{\pi^{2}- 8}{16}
    \]
\end{ex}

\subsection{Fourier Transform as limit of Fourier Series}
We discussed $2\pi$-periodic functions in detail. 
Consider now $f(x)$ being periodic on $[-L, L]$ with $L$ being arbitrary.
We have shown that \[
    f(x) = \sum_{n=-\infty}^{\infty} r_n e^{in \frac{\pi x}{L}}, \, -L \le x \le L
\]where\[
r_n = \frac{1}{2L}\int_{-L}^{L} f(t) e^{-in \frac{\pi t}{L}} \mathrm{d}t, \, n = 0, \pm 1, \pm 2, \ldots
\]
Put $r_n$ into the sum to find\[
    f(x) = \sum_{n=-\infty}^{\infty} \left(\frac{1}{2L} 
    \int_{-L}^{L} f(t) e^{-in \frac{\pi t}{L}} \mathrm{d}t\right) e^{in \frac{\pi x}{L}}
\]
This is exact. We want to send $L \rightarrow \infty$.\[
    f(x) = \frac{1}{2\pi}\sum_{n=-\infty}^{\infty} h\left(
    \int_{-L}^{L} f(t)e^{-inht} \mathrm{d}t\right) e^{inhx}
\]
where $h = \frac{\pi}{L}$.
In the limit $L \rightarrow \infty$, $h \rightarrow \infty$, 
but $nh := \omega_n = 0(1)$.\[
    f(x) = \frac{1}{2\pi} \sum_{n=-\infty}^{\infty} h\left(
    \int_{-L}^{L} f(t)e^{-i\omega_n t} \mathrm{d}t\right)e^{i\omega_n x}
\]
This is of the form $\sum_{n=-\infty}^{\infty} G(\omega_n)h$.
Now $h = \omega_{n+1} - \omega_n = (n + 1)h - nh := \delta\omega$,\[
    \text{Riemann sum} \quad \sum_{n=-\infty}^{\infty} G(\omega_n)\delta\omega 
    \rightarrow \int_{-\infty}^{\infty} G(\omega_n)\mathrm{d}\omega
\]
This gives, sending $L \rightarrow \infty$, \[
    f(x) = \frac{1}{2\pi}\int_{-\infty}^{\infty} \left(\int_{-\infty}^{\infty} 
    f(t) e^{-i\omega t}\mathrm{d}t\right) e^{i\omega x} \mathrm{d}\omega
\]where $f(x)$ is defined on $\mathbb{R}$.
This gives the \emph{Fourier Transform pair}:
\medskip
\\\fbox{\begin{minipage}{\linewidth}
\[
    f(x) = \frac{1}{2\pi}\int_{-\infty}^{\infty} \hat{f}(k) e^{ikx}\mathrm{d}k
\]\[
\hat{f}(x) = \int_{-\infty}^{\infty} f(x) e^{-ikx}\mathrm{d}x
\]
\end{minipage}}
\medskip
\\This is very useful in many applications. You will use them a lot to solve differential equations.



\end{document} %chktex 17
