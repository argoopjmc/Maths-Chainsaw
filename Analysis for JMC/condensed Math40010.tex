\documentclass[12pt]{report}
\usepackage[thinc]{esdiff} % for typesettign derivatives
\usepackage{amsthm} % provides an enhanced version of LaTex's \newtheorem command
\usepackage{mdframed} % framed environments that can split at page boundaries
\usepackage{enumitem} % bulletin points or other means of listing things
\usepackage{amssymb} % for AMS symbols

\theoremstyle{definition}
\mdfdefinestyle{defEnv}{%
  hidealllines=false,
  nobreak=true,
  innertopmargin=-1ex,
}

\pagestyle{headings}
\author{Aris Zhu Yi Qing}
\title{Condensed Notes for Maths40010}
\begin{document}
\maketitle
\tableofcontents

\chapter{Numbers}

\section{Countability}
\newmdtheoremenv[style=defEnv]{theorem}{Definition}
\begin{theorem}
    A set $S$ is \emph{countable} iff $\exists$ bijection $f: \mathbb{N} \rightarrow S$.
\end{theorem}

\newmdtheoremenv[style=defEnv]{subset of countable set}[theorem]{Theorem}
\begin{subset of countable set}
    Suppose $S \subset \mathbb{N}$ is infinite. Then $S$ is \emph{countable}.
\end{subset of countable set}

\newmdtheoremenv[style=defEnv]{Z is countable}[theorem]{Theorem}
\begin{Z is countable}
    \mathbb{Z} is countable.
\end{Z is countable}

\newmdtheoremenv[style=defEnv]{Q is countable}[theorem]{Theorem}
\begin{Q is countable}
    \mathbb{Q} is countable.
\end{Q is countable}

\newmdtheoremenv[style=defEnv]{R is uncountable}[theorem]{Theorem}
\begin{R is uncountable}
    \mathbb{R} is uncountable.
\end{R is uncountable}

\section{The Completeness Axiom}
\newmdtheoremenv[style=defEnv]{upper bound and sup}[theorem]{Definition}
\begin{upper bound and sup}
    $\varnothing \neq S \subset \mathbb{R}$ is \emph{bounded above} if \[
        \exists M \in \mathbb{R} \,\textnormal{s.t.} \, \forall x \in S, x \le M
    \]
    Such an M is called an \emph{upper bound} for $S$. 
    In addition, we say $x \in \mathbb{R}$ is a \emph{least upper bound} for $S$ or \textbf{supremum} of $S$ iff
    \begin{itemize}
        \item $x$ is an upper bound for $S$ (i.e. $x \ge s \;\forall s \in S$)
        \item $x \le y \;\forall$ upper bounds $y$ of $S$ (i.e. $y \ge s \;\forall s \in S \Rightarrow y \ge x$)
    \end{itemize}
\end{upper bound and sup}

\newmdtheoremenv[style=defEnv]{lower bound and inf}[theorem]{Theorem}
\begin{lower bound and inf}
    $\varnothing \neq S \subset \mathbb{R}$ is \emph{bounded below} if \[
        \exists M \in \mathbb{R} \,\textnormal{s.t.} \, \forall x \in S, x \ge M
    \]
    Such an M is called an \emph{lower bound} for $S$. 
    In addition, we say $x \in \mathbb{R}$ is a \emph{greatest lower bound} for $S$ or \textbf{infimum} of $S$ iff
    \begin{itemize}
        \item $x$ is a lower bound for $S$ (i.e. $x \le s \;\forall s \in S$)
        \item $x \ge y \;\forall$ lower bounds $y$ of $S$ (i.e. $y \le s \;\forall s \in S \Rightarrow y \le x$)
    \end{itemize}
\end{lower bound and inf}

\newmdtheoremenv[style=defEnv]{Completeness axiom of real numbers}[theorem]{Theorem}
\begin{Completeness axiom of real numbers}
    Suppose $S \subseteq \mathbb{R}$ is nonempty, bounded above, then $\exists \sup{S} \in \mathbb{R}$
\end{Completeness axiom of real numbers}

\section{Dedekind cuts}
\newmdtheoremenv[style=defEnv]{Dedekind cut}[theorem]{Definition}
\begin{Dedekind cut}
    We say a nonempty subset $s \subset \mathbb{Q}$ is a \emph{Dedekind cut} if it satisfy
    \begin{enumerate}[label = (\roman*)]
        \item $\forall s \in S$, $[t < s \Rightarrow t \in S]$, i.e. $S$ is a semi-infinite interval to the left.
        \item $S$ has an upper bound but \underline{no maximum}
    \end{enumerate}
\end{Dedekind cut}

\newmdtheoremenv[style=defEnv]{Dedekind cut's version of real numbers}[theorem]{Definition}
\begin{Dedekind cut's version of real numbers}
    New Definition of $\mathbb{R}$:\[
    \mathbb{R} := \{ \textnormal{Dedekind cuts} \;S \subset \mathbb{Q}\}
\]
\end{Dedekind cut's version of real numbers}

\section{triangle inequalities}
\newmdtheoremenv[style=defEnv]{basic triangle ineq}[theorem]{Theorem}
\begin{basic triangle ineq}
    $\forall a, b \in \mathbb{R}$, we have\[
        |a + b| \le |a| + |b|
    \]\[
        |a + b| \ge \Big||a| - |b|\Big|
    \]\[
        |a| \le |b| + |a - b|
    \]\[
        |a| \ge |b| - |a - b|
    \]\[
        |a - b| \le |a - c| + |b - c|
    \]
\end{basic triangle ineq}

\chapter{Sequences}
\newmdtheoremenv[style=defEnv]{sequence}[theorem]{Definition}
\begin{sequence}
    A \emph{sequence} is a function $a : \mathbb{N} \rightarrow \mathbb{R}$
\end{sequence}

\section{convergence of sequences}
\newmdtheoremenv[style=defEnv]{convergence}[theorem]{Definition}
\begin{convergence}
    We say that $a_n \rightarrow a$ as $n \rightarrow \infty$ iff \[
        \forall \varepsilon > 0, \exists N \in \mathbb{N} \;\textnormal{s.t.} \; |a_n - a| < \varepsilon \;\forall n > N
    \]
\end{convergence}

\newmdtheoremenv[style=defEnv]{convergent sequence}[theorem]{Definition}
\begin{convergent sequence}
    We say that $a_n$ \emph{converges} iff $\exists a \in \mathbb{R}$ s.t. $a_n \rightarrow a$, i.e. $a_n$ converges iff\[
        \exists a \in \mathbb{R} \;\textnormal{s.t.}\; \forall \varepsilon > 0, \exists N \in \mathbb{N} 
        \;\textnormal{s.t.}\; \forall n \ge N, |a_n - a| < \varepsilon
    \]
\end{convergent sequence}

\newmdtheoremenv[style=defEnv]{divergent sequence}[theorem]{Definition}
\begin{divergent sequence}
    We say $a_n$ \emph{diverges} iff it does not converge (to any $a \in \mathbb{R}$), i.e. \[
        \forall a \in \mathbb{R}, \exists \varepsilon > 0 \;\textnormal{s.t.}\; \forall N \in \mathbb{N},
        \exists n \ge N \;\textnomral{s.t.}\; |a_n - a| \ge \varepsilon
    \]
\end{divergent sequence}

\newmdtheoremenv[style=defEnv]{approach to infinity}[theorem]{Definition}
\begin{approach to infinity}
    We say $a_n \rightarrow +\infty$ iff \[
        \forall R > 0, \exists N \in \mathbb{N} \;\textnormal{s.t.}\; \forall n \ge N, a_n > R
    \]
\end{approach to infinity}


\newmdtheoremenv[style=defEnv]{convergent complex sequence}[theorem]{Definition}
\begin{convergent complex sequence}
    $a_n \in \mathbb{C}, \forall \ge 1$. We say $a_n \rightarrow a \in \mathbb{C}$ iff \[
        \forall \varepsilon > 0, \exists N \in \mathbb{N} \;\textnormal{s.t.}\; n \ge N 
        \Rightarrow |a_n - a| < \varepsilon
    \]
\end{convergent complex sequence}

\newmdtheoremenv[style=defEnv]{uniqueness of limits}[theorem]{Theorem}
\begin{uniqueness of limits}
    Limits are unique. If $a_n \rightarrow a$ and $a_n \rightarrow b$, then $a = b$.
\end{uniqueness of limits}

\newmdtheoremenv[style=defEnv]{algebra of limits}[theorem]{Theorem}
\begin{algebra of limits}
    if $a_n \rightarrow a$ and $b_n \rightarrow b$ then:
    \begin{enumerate}
        \item $a_n + b_n \rightarrow a + b$
        \item $a_n b_n \rightarrow ab$
        \item $\frac{a_n}{b_n} \rightarrow \frac{a}{b}$ if $b \neq 0$.
    \end{enumerate}
\end{algebra of limits}

\newmdtheoremenv[style=defEnv]{monotonically increasing to converge}[theorem]{Theorem}
\begin{monotonically increasing to converge}
    If $(a_n)$ is \emph{bounded above} and \emph{monotonically increasing} then $a_n$ converges to
    $a := \sup{\{a_i : i \in \mathbb{N}\}}$. We write $a_n \uparrow a$.
\end{monotonically increasing to converge}

\section{Cauchy Sequences}
\newmdtheoremenv[style=defEnv]{cauchy sequences}[theorem]{Definition}
\begin{cauchy sequences}
    ${(a_n)}_{n \ge 1}$ is called a \emph{Cauchy} sequence iff \[
        \forall \varepsilon > 0, \exists N \in \mathbb{N} \;\textnormal{s.t.}\; 
        \forall n,m \ge N, |a_n - a_m| < \varepsilon
    \]
\end{cauchy sequences}

\newmdtheoremenv[style=defEnv]{cauchy sequence is bounded}[theorem]{Theorem}
\begin{cauchy sequence is bounded}
    $(a_n)$ is Cauchy $\Rightarrow$ $(a_n)$ is bounded.
\end{cauchy sequence is bounded}

\newmdtheoremenv[style=defEnv]{cauchy sequence is convergenct}[theorem]{Theorem}
\begin{cauchy sequence is convergenct}
    $(a_n)$ is Cauchy $\iff$ $(a_n)$ is convergent.
\end{cauchy sequence is convergenct}

\section{Subsequences}
\newmdtheoremenv[style=defEnv]{subsequence}[theorem]{Definition}
\begin{subsequence}
    A \emph{subsequence} of $a_n$ is a new sequence $b_i = a_{n(i)} \forall i \in \mathbb{N}$,
    where $n(1) < n(2) < \cdots < n(i) < \ldots \forall i$.
\end{subsequence}

\newmdtheoremenv[style=defEnv]{Bolzno-Weierstrass}[theorem]{Theorem}
\begin{Bolzno-Weierstrass}
    If $(a_n)$ is a \emph{bounded} sequence of real numbers, then it has a convergent subsequence.
\end{Bolzno-Weierstrass}

\newmdtheoremenv[style=defEnv]{subsequence converge to the same value}[theorem]{Theorem}
\begin{subsequence converge to the same value}
    If $a_n \rightarrow a$ then any subsequence $a_{n(i)} \rightarrow a$ as $i \rightarrow \infty$.
\end{subsequence converge to the same value}

\section{Series}
\newmdtheoremenv[style=defEnv]{series}[theorem]{Definition}
\begin{series}
    An (infinite) series is an expression \[
        \sum_{n=1}^{\infty} a_n \quad \textnormal{or} \quad a_1 + a_2 + a_3 + \ldots,
    \]where ${(a_i)}_{i \ge 1}$ is a sequence.
\end{series}

\newmdtheoremenv[style=defEnv]{partial sum}[theorem]{Definition}
\begin{partial sum}
    $n^{\text{th}}$ partial sum is\[
        S_n := \sum_{i=1}^{n} a_i \in \mathbb{R}
    \]
\end{partial sum}

\section{Convergence of Series}
\newmdtheoremenv[style=defEnv]{convergent series}[theorem]{Definition}
\begin{convergent series}
    We say that the series $\sum a_n$ ``converges to $A \in \mathbb{R}$'' iff the sequence $(S_n)$ of partial sums
    converges to $A$:\[
        \sum_{n=1}^{\infty} a_n = A \in \mathbb{R} \iff S_n \rightarrow A \;\textnormal{as $n \rightarrow\infty$}
    \]
\end{convergent series}

\newmdtheoremenv[style=defEnv]{convergent series' term tends to 0}[theorem]{Theorem}
\begin{convergent series' term tends to 0}
    $\sum_{n=1}^{\infty} a_n$ is convergent $\Rightarrow a_n \rightarrow 0$. In other words, 
    $a_n \nrightarrow 0 \Rightarrow \sum_{n=1}^{\infty} a_n$ is divergent.
\end{convergent series' term tends to 0}

\newmdtheoremenv[style=defEnv]{determine convergent series}[theorem]{Theorem}
\begin{determine convergent series}
    Suppose $a_n \ge 0 \; \forall n$ ($\iff S_n = \sum_{i=1}^{n} a_i$ monotonically increasing), 
    then $S_\infty = \sum_{n=1}^{\infty} a_n$ convergent $\iff$ $(S_n)$ bounded above.
    Similary, $\sum_{n=1}^{\infty} a_n \rightarrow +\infty \iff (S_n)$ is unbounded.
\end{determine convergent series}

\newmdtheoremenv[style=defEnv]{Algebra of limits for series}[theorem]{Theorem}
\begin{Algebra of limits for series}
    if $\sum a_n$, $\sum b_n$ are convergent then so is $\sum (\lambda a_n + \mu b_n)$, to\[
        \sum_{n=1}^{\infty} (\lambda a_n + \mu b_n) = \lambda \sum_{n=1}^{\infty} a_n + \mu \sum_{n=1}^{\infty} b_n
    \]
\end{Algebra of limits for series}

\section{Absolute convergence}
\newmdtheoremenv[style=defEnv]{absolutely convergent}[theorem]{Definition}
\begin{absolutely convergent}
    For $a_n \in \mathbb{R}$ or $\mathbb{C}$, we say the series $\sum_{n=1}^{\infty} a_n$ is \emph{absolutely convergent}
    iff the series $\sum_{n=1}^{\infty} |a_n|$ is convergent.
\end{absolutely convergent}

\newmdtheoremenv[style=defEnv]{absolute convergence is convergence}[theorem]{Theorem}
\begin{absolute convergence is convergence}
    If $\sum a_n$ is absolutely convergent, then it is convergent.
\end{absolute convergence is convergence}

\section{Tests for convergence}
\newmdtheoremenv[style=defEnv]{comparison test 1}[theorem]{Theorem}
\begin{comparison test 1}
    if $0 \le a_n \le b_n$ and $\sum_{n=1}^{\infty} b_n$ is convergent, then
    $\sum a_n$ convergent and $0 \le \sum a_n \le \sum b_n$.
\end{comparison test 1}

\newmdtheoremenv[style=defEnv]{comparison test 2}[theorem]{Theorem}
\begin{comparison test 2}
    If $c_n \le a_n \le b_n \;\forall n$ and $\sum c_n$, $\sum b_n$ both convergent, 
    then $\sum a_n$ convergent and $\sum c_n \le \sum a_n \le \sum b_n$.
\end{comparison test 2}

\newmdtheoremenv[style=defEnv]{comparison test 3}[theorem]{Theorem}
\begin{comparison test 3}
    If $\frac{a_n}{b_n} \rightarrow L \in \mathbb{R} (b_n \neq 0 \;\forall n)$,
    then if $\sum b_n$ is absolutely convergent, then $\sum a_n$ is absolutely convergent.
\end{comparison test 3}

\newmdtheoremenv[style=defEnv]{alternating series test}[theorem]{Theorem}
\begin{alternating series test}
    If $(a_n)$ is alternating and $|a_n| \downarrow 0$, then $\sum a_n$ is convergent.
\end{alternating series test}

\newmdtheoremenv[style=defEnv]{ratio test}[theorem]{Theorem}
\begin{ratio test}
    If $\left|\frac{a_{n+1}}{a_n}\right| \rightarrow r < 1$, then $\sum a_n$ is absolutely convergent.
\end{ratio test}

\newmdtheoremenv[style=defEnv]{root test}[theorem]{Theorem}
\begin{root test}
    If ${|a_n|}^{\frac{1}{n}} \rightarrow r < 1$, then $\sum a_n$ is absolutely convergent.
\end{root test}

\section{Rearangement of series}
\newmdtheoremenv[style=defEnv]{Rearrangement of a sequence}[theorem]{Definition}
\begin{Rearrangement of a sequence}
    Given a bijection $n: \mathbb{N} \rightarrow \mathbb{N}$, define $b_i := a_{n(i)}$. 
    Then ${(b_i)}_{i \ge 1}$ is a \emph{rearrangement} or \emph{reordering} of ${(a_n)}_{n \ge 1}$.
\end{Rearrangement of a sequence}

\newmdtheoremenv[style=defEnv]{convergence of reordered series}[theorem]{Theorem}
\begin{convergence of reordered series}
    $\sum a_n$ is absolutely convergent $\iff (1) + (2) \Rightarrow (3) + (4)$, where
    \begin{enumerate}
        \item $\sum_{a_n \ge 0}^{} a_n$ is convergent (to $A$ say),
        \item $\sum_{a_n < 0}^{} a_n$ is convergent (to $B$ say),
        \item $\sum a_n = A + B$,
        \item $\sum b_m = A + B$ where $(b_m)$ is any rearrangement of $(a_n)$.
    \end{enumerate}
\end{convergence of reordered series}

\section{Power Series}
\newmdtheoremenv[style=defEnv]{radius of convergent}[theorem]{Theorem}
\begin{radius of convergent}
    Fix a real complex series $(a_n)$ an consider the series $\sum a_n z^{n}$ for $z \in \mathbb{C}$.
    Then $\exists R \in [0, \infty]$ s.t. 
    \begin{itemize}
        \item $|z| < R \Rightarrow \sum a_n z^{n}$ is absolutely convergent, and
        \item $|z| > R \Rightarrow \sum a_n z^{n}$ is divergent.
    \end{itemize}
\end{radius of convergent}

\subsection{Products of Series}
\newmdtheoremenv[style=defEnv]{cauchy product}[theorem]{Definition}
\begin{cauchy product}
    Given series $\sum a_n$, $\sum b_n$, their \emph{Cauchy Product} is the series $\sum c_n$
    where $c_n := \sum_{i=0}^{n} a_i b_{n-i}$.
\end{cauchy product}

\newmdtheoremenv[style=defEnv]{Cauchuy Product convergence}[theorem]{Theorem}
\begin{Cauchuy Product convergence}
    If $\sum a_n$, $\sum b_n$ are absolutely convergent, then their Cauchy Product $\sum c_n$ 
    is absolutely convergent to $\left(\sum a_n\right) \cdot \left(\sum b_n\right)$.
\end{Cauchuy Product convergence}

\section{Exponential Power Series}
\newmdtheoremenv[style=defEnv]{exp series}[theorem]{Definition}
\begin{exp series}
    For any $z  \in \mathbb{C}$ set \[
        E(z) := \sum_{n=0}^{\infty} \frac{z^{n}}{n!}
    \]
\end{exp series}

\newmdtheoremenv[style=defEnv]{exp series properties}[theorem]{Theorem}
\begin{exp series properties}
    $E(x)$ has the following properties for $x \in \mathbb{R}$.
    \begin{enumerate}
        \item $E(x) > 0 \;\forall x \in \mathbb{R}$
        \item $x \ge 0 \Rightarrow E(x) \ge 1$ and $x > 0 \Rightarrow E(x) > 1$
        \item $E(x)$ is strictly increasing for $x \in \mathbb{R}$
        \item $|E(x) - 1| < \frac{|x|}{1 - |x|} \;\forall |x| < 1$
        \item $x \mapsto E(x)$ is a continuous bijection $\mathbb{R} \rightarrow (0, \infty)$
    \end{enumerate}
\end{exp series properties}

\chapter{Continuity}
\section{Limits}
\newmdtheoremenv[style=defEnv]{limit}[theorem]{Definition}
\begin{limit}
    Fix a function $f:\mathbb{R} \rightarrow \mathbb{R}$ and points $a,b \in \mathbb{R}$.
    \\We say that $f(x) \rightarrow b$ as $x \rightarrow a$ (or ``$\lim_{x\rightarrow a} f(x) = b$'') iff\[
        \forall \varepsilon > 0, \exists \delta > 0 \;\textnormal{s.t.}\; 0 < |x-a| < \delta
        \Rightarrow |f(x) - b| < \varepsilon
    \]
\end{limit}

\section{Continuity}
\newmdtheoremenv[style=defEnv]{continuous function}[theorem]{Definition}
\begin{continuous function}
    Given a function $f:\mathbb{R} \rightarrow \mathbb{R}$, we say that $f$ is \emph{continuous} at $a \in \mathbb{R}$ 
    iff \[
        \forall \varepsilon > 0, \exists \delta > 0 \;\textnormal{s.t.}\; |x-a|<\delta \Rightarrow
        |f(x) - f(a)| < \varepsilon
    \]
    We say that $f$ is continuous on $\mathbb{R}$ (or just ``continuous'') if it is continuous at all $a \in \mathbb{R}$.
\end{continuous function}

\newmdtheoremenv[style=defEnv]{continuity with convergent sequence}[theorem]{Theorem}
\begin{continuity with convergent sequence}
    For all sequences $(x_n)$ which tends to $a$, 
    $f:\mathbb{R} \rightarrow \mathbb{R}$ is continuous at $a \in \mathbb{R} 
    \iff f(x_n) \rightarrow f(a)$
\end{continuity with convergent sequence}


\end{document}
